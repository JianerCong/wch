% \documentclass[dvipsnames]{article}
\documentclass[dvipsnames]{ctexart}
\title{存储执行模块开发文档1.0}
\usepackage{geometry}\geometry{
  a4paper,
  total={170mm,257mm},
  left=20mm,
  top=20mm,
}


\usepackage{adjustbox}          %to narrower the caption
\newlength\mylength
\usepackage{hyperref}
\usepackage{svg}
\usepackage[skip=5pt plus1pt, indent=0pt]{parskip}
\usepackage{emoji}
% \setemojifont{NotoColorEmoji.ttf}[Path=C:/Users/congj/repo/myFonts/]
% \setemojifont{TwitterColorEmoji-SVGinOT.ttf}[Path=C:/Users/congj/repo/myFonts/]

\usepackage{booktabs}
\usepackage{tikz}
\usetikzlibrary{shapes} % ellipse node shape
\usetikzlibrary{shapes.multipart} % for line breaks in node text
\usetikzlibrary{arrows.meta}    %-o arrow head
\usetikzlibrary{arrows}
\usetikzlibrary{matrix}

\usepackage{amsmath}

\usepackage{minted}
\usepackage{tcolorbox}
\tcbuselibrary{skins}
\tcbuselibrary{minted}

\usepackage{fontspec}
\setmonofont{Cascadia}[
Path=/usr/share/fonts/truetype/Cascadia_Code/,
Scale=0.85,
Extension = .ttf,
UprightFont=*Code,              %find CascadiaCode.ttf
BoldFont=*CodePL,               %find CascadiaCodePL.ttf ...
ItalicFont=*CodeItalic,
BoldItalicFont=*CodePLItalic
]

% --------------------------------------------------
% Windows
% \setmonofont{Cascadia}[
% Path=C:/Windows/Fonts/,
% Extension = .ttf,
% UprightFont=*Mono,              %find CascadiaMono.ttf
% BoldFont=*Code,               %find CascadiaCodePL.ttf ...
% ItalicFont=*Code,
% BoldItalicFont=*Code
% ]


% Without this, there's a gap between the upper bbox and the node box

\newtcblisting{simplec}{
  listing engine=minted,
  minted language=c++,
  minted style=vs,
  minted options={fontsize=\small,autogobble,
  % framesep=1cm
  },
  tile,
  listing only,
  % bottom=0cm,
  % nobeforeafter, 
  boxsep=0mm,
  left=1mm,
  opacityback=0.5,
  colback=gray!20
}

\newcounter{myCListing}
\newtcblisting{numberedc}[3][]{
  listing engine=minted,
  minted language=c++,
  minted style=vs,
  minted options={fontsize=\small,autogobble,
  % framesep=1cm
  },
  tile,
  listing only,
  % bottom=0cm,
  % nobeforeafter, 
  boxsep=1.5mm,
  left=1mm,
  opacityback=0.5,
  colback=gray!20,
  phantom={
  \refstepcounter{myCListing}
  #3
  },
  title={代码~\themyCListing: #2},
  #1
}


\newtcblisting{simplepy}{
  listing engine=minted,
  minted language=python,
  minted style=vs,
  minted options={fontsize=\small,autogobble},
  tile,
  listing only,
  % nobeforeafter, 
  boxsep=0.5mm,
  left=1mm,
  opacityback=0.5,
  colback=gray!20,
}


\tcbset{enhanced, fontupper=\small,
  % show bounding box
}

% Color
\newcommand{\mycola}{MidnightBlue}
\newcommand{\mycolb}{Mahogany}
\newcommand{\mycolc}{OliveGreen}

\newcommand{\cola}[1]{\textcolor{\mycola}{#1}}
\newcommand{\colb}[1]{\textcolor{\mycolb}{#1}}
\newcommand{\colc}[1]{\textcolor{\mycolc}{#1}}
\newcommand{\Cola}[1]{\textcolor{\mycola}{\emph{#1}}}


\newcommand{\myTwo}[2]{\texttt{#1}为\texttt{#2}}

\tcbset{right=1mm,halign=flush left}
\newtcolorbox{ifaceBox}[1][]{colframe=\mycola, nobeforeafter, 
  fontupper=\footnotesize, left=2mm, #1}

\newtcolorbox{varBox}[1][]{nobeforeafter, 
  fontupper=\footnotesize, left=2mm, #1}

\newtcolorbox{weakBox}[1][]{colframe=gray!80, nobeforeafter,
  fontupper=\footnotesize, left=2mm, #1}

\newtcolorbox{funcBox}[1][]{colframe=\mycolb, nobeforeafter, 
  fontupper=\footnotesize, left=2mm, #1}

\newtcolorbox{noteBox}[1][]{colframe=\mycolc, nobeforeafter, 
  fontupper=\footnotesize, left=2mm, #1}

\newtcbox{\libbox}[1][\mycala]{on line,
  arc=0pt,outer arc=0pt,colback=#1!10!white,colframe=#1!50!black,
  boxsep=0pt,left=1pt,right=1pt,top=2pt,bottom=2pt,
  boxrule=0pt,bottomrule=1pt,toprule=1pt}

% copied from tcolorbox mannual
\newtcbox{\mylib}[1][\mycola]{on line,
  arc=0pt,outer arc=0pt,colback=#1!10!white,colframe=#1!50!black,
  boxsep=0pt,left=1pt,right=1pt,top=2pt,bottom=2pt,
  boxrule=0pt,bottomrule=1pt,toprule=1pt,
  left skip=0.1cm,
  right skip=0.1cm,
  fontupper=\ttfamily
}


% Redefine em
%latex.sty just do: \DeclareTextFontCommand{\emph}{\em}

\let\emph\relax % there's no \RedeclareTextFontCommand
\DeclareTextFontCommand{\emph}{\bfseries\em}

% Redefine section
\usepackage[small]{titlesec}


% curve version
% \newcommand\uptoleft[2][]{\draw[very thick,->](#1.south) to [out=270,in=180] (#2.west);}

% right-angle version
\newcommand\Southtowest[3][-o]{\draw[very thick,#1](#2.south) |- (#3.west);}
\newcommand\Southtoeast[3][-o]{\draw[very thick,#1](#2.south) |- (#3.east);}
\newcommand\Northtowest[3][-o]{\draw[very thick,#1](#2.north) |- (#3.west);}
\newcommand\Northtoeast[3][-o]{\draw[very thick,#1](#2.north) |- (#3.east);}

\newcommand\southtonorth[3][-o]{\draw[very thick,#1](#2.south) to [out=270,in=90] (#3.north);}
\newcommand\northtosouth[3][-latex]{\draw[very thick,#1](#2.north) to [out=90,in=270] (#3.south);}


\newcommand\easttowest[3][-latex]{\draw[very thick,#1](#2.east) to[out=0,in=180] (#3.west);}
\newcommand\easttonorth[3][-latex]{\draw[very thick,#1](#2.east) to[out=0,in=90] (#3.north);}
\newcommand\easttosouth[3][-latex]{\draw[very thick,#1](#2.east) to[out=0,in=270] (#3.south);}

% the left node link my ``up''

\newcommand\westTowest[3][-latex]{\draw[very thick,#1](#2.west) to[out=180,in=180] (#3.west);}
\newcommand\westTonorth[3][-latex]{\draw[very thick,#1](#2.west) to[out=180,in=90] (#3.north);}
\newcommand{\mySection}[1]{\section*{\texttt{#1}}}

\tikzstyle{every node}=[inner sep=0pt]
\tikzstyle{iface}=[rectangle,fill=gray!30,inner sep=2mm]
\tikzstyle{myMatrix}=[matrix of nodes,below right,
nodes={right},                  %apply to all nodes
row sep=1cm,column sep=2cm]

% TREE NODE
\tikzstyle{myTreeNode}=[draw=\mycola,anchor=west,inner sep=2mm,rectangle,rounded corners,fill=gray!20]
\tikzstyle{edge from parent}=[draw=\mycola,thick,-latex]
\tikzstyle{every child node}=[myTreeNode]

\usepackage{simpsons}
\usepackage{cleveref}
\crefname{figure}{图}{图}
%                    ^^^ plural
\Crefname{figure}{图}{图}
%         ^^^^^^ type = counter name

\crefname{table}{表}{表}
\Crefname{table}{表}{表}

\crefname{section}{}{}
\Crefname{section}{}{}
\creflabelformat{section}{第#2#1#3章节}


% #2 : start of hyperlink
% #3 : end of hyperlink
% #1 : The counter

% U need to define both crefname and Crefname , else it will cause fatal error.
\crefname{myCListing}{代码}{代码}
\Crefname{myCListing}{代码}{代码}
\date{\today}
\author{新华夏技术中心}
\tcbuselibrary{breakable}
\newtcolorbox{blackbox}{tile,colback=black,colupper=white,nobeforeafter,halign=flush center}

\newcommand{\mycolaa}{\mycola!20}

\usepackage{placeins}

% \usepackage{lscape}             %for landscape env
\usepackage{pdflscape} %uncomment this  and comment above line to see the difference
% --------------------------------------------------

\usepackage{tabularx}
\begin{document}
\maketitle
\tableofcontents{}
\newpage{}


\section{摘要}

这篇文档介绍存储和执行模块。我们将从抽象到形象,底层到上层,依次介绍接口和类。

\section{区块链世界存储}
\label{sec:general-world-storage}
广义上来讲,区块链的存储就是两个部分:
\begin{enumerate}
\item 一个 \cola{账本} 和
\item 一个 \colb{状态}
\end{enumerate}
\cola{账本}里面(最主要地)记录了\textbf{已经执行了
  的\colc{交易}},而\colb{状态}里面记录了\textbf{当前状态},这些状态因\colc{交易}的
执行而改变,如\cref{fig:worldStorage} 。


\begin{figure}[h]
  \begin{center}
    \begin{tikzpicture}
      % \draw[style=help lines] (0cm,0cm) grid +(15cm,-10cm);
      \node (WorldStorage) [text width=6cm] at (2cm,-2cm)
      {\begin{ifaceBox}[]
          % \big
          \large
          \textbf{区块链数据World Storage}
        \end{ifaceBox}};
      
      \matrix (M-WorldStorage) [myMatrix,
      shift={(2,-2)},
      text width=12cm]
      at (WorldStorage.center){
        \raisebox{-.3\height}{\includesvg[width=0.5cm]{img/database.svg}}
        % ------- the height of the to-be raised box
        \cola{账本}: 重点包括一个个已经执行了的区块 (“整个链”),以及其他信息。因
        此有些时候我们也说 “\cola{链} + \colb{状态}” 而不是 “\cola{账本} + \colb{状态}”
        \\
        \raisebox{-.3\height}{\includesvg[width=0.5cm]{img/database.svg}} \colb{状
         态}: 当前\colb{世界状态 World State},每次区块数增加时被改变。
        \\
      }; \foreach \i in {1,2}{ \Southtowest{WorldStorage}{M-WorldStorage-\i-1} }
    \end{tikzpicture}
  \end{center}
  \caption{广义的区块链世界数据}
  \label{fig:worldStorage}
\end{figure}

而这个\colb{世界状态}其实就是一个KV表,见\cref{fig:worldState}其中
\begin{itemize}
\item Key 为 \colc{地址 Address}, 20-bytes 字符串
\item Value 为 \cola{账户 Account},象征着每一个每一个部署的合约或用户。
\end{itemize}
\begin{figure}[h]
  \begin{center}
    \begin{ifaceBox}[width=\linewidth/2,title=\texttt{WorldState}]
      \begin{center}
        \begin{tikzpicture}
          \matrix (M1) [matrix of nodes,column sep=2cm, row sep=0.3cm, nodes={ellipse,
            text height=1.3em,
            text depth=1.1em,
            text centered
          }]{
            |[draw,text=\mycolc]| \texttt{address-1} & |[draw,text=\mycola]| \texttt{account-1} \\
            |[draw,text=\mycolc]| \texttt{address-2} & |[draw,text=\mycola]| \texttt{account-2} \\
            |[draw,text=\mycolc]| \texttt{address-3} & |[draw,text=\mycola]| \texttt{account-3} \\
            \texttt{...} & \texttt{...} \\
          };
          \foreach \i in {1,2,3}{
            \easttowest{[xshift=.5cm]M1-\i-1}{[xshift=-.5cm]M1-\i-2}
          }
        \end{tikzpicture}
      \end{center}
    \end{ifaceBox}
  \end{center}
  \caption{世界状态World State内部是address到account的KV表}
  \label{fig:worldState}
\end{figure}

\section{底层数据结构}

\subsection{账户 Account}

而这里我们就要介绍非常重要的\cola{账户Account}数据结构。
\begin{tcolorbox}
  \emoji{parrot} : 要按重要性把区块链世界里的数据结构来排名的
  话,我觉得\cola{账户Account}起码能排前三。
\end{tcolorbox}
Account,除了一些其他辅助类信息(如\texttt{nonce},\texttt{hash},...),重点包含以
下两个数据:
\begin{itemize}
\item \texttt{storage} 存储: 一个\texttt{bytes32} 到 \texttt{bytes32}的 KV 表。合
  约会往这里读写数据。
\item \texttt{code} 代码: 该账户上的字节码,应该是一个可以被塞进虚拟机里执行的。
\end{itemize}
账户分为两种:
\begin{enumerate}
\item \cola{合约账户} : 代表着一个被部署了的合约,其中\texttt{code}为调用该合约的
  字节码。该字解码要配合一个ABI字符串来使用(以后说)。
\item  \colb{非合约账户}: 一般代表一个用户,没有必要有\texttt{code}。
\end{enumerate}

\begin{tcolorbox}
  \emoji{parrot} : 其实传统的Account里都还会有一个\texttt{value}代表着账户余额,
  但我们不需要,也就不提了。
\end{tcolorbox}

那么我们来看代码, 见\cref{code:acn}\footnote{上面
  的\texttt{IJsonizable}和\texttt{ISerializable}是两个描述行为(behavior)的虚接
  口。对于了解我们的数据结构而言并不重要,暂时可以先不理解我们以后再说(见\cref{sec:iface})。不过如果
  好奇的话,大概意思就是说:“这个类可以变成JSON”,“这个类可以被序列化/反序列化”。}

\begin{numberedc}{账户}{\label{code:acn}}
class Acn: virtual public IJsonizable
         , virtual public ISerializable{
public:
  Acn() = default;
  Acn(const uint64_t n,bytes c): nonce(n), code(c){
    // get hash from code
    codehash = ethash::keccak256(reinterpret_cast<const uint8_t*>(c.data()),
                                 c.size());
  };
  uint64_t nonce = 0; /// The account nonce.
  bytes code;
  hash256 codehash;
  unordered_map<bytes32,bytes32> storage;
  //... 略
}
\end{numberedc}

\begin{tcolorbox}
  \emoji{turtle} : 以上所有的类都来自于\texttt{evmc}和\texttt{ethash}:
  \begin{simplec}
    using evmc::bytes;
    using evmc::bytes_view;
    using evmc::address;
    using evmc::bytes32;
    using ethash::hash256;
  \end{simplec}
  使用模块现有的类是为了最小化代码量(也最小化bug可能性)。
\end{tcolorbox}
可以看到我们现在最简化的Account里只有4个成员:
\texttt{nonce,code,codehash,storage}。只需要这些成员就可以和EVMC进行交互了。

如果把每一个Account想象成一个小U盘的话,里面的\cola{\texttt{code}}就是里面
的\cola{程序},而里面的\colc{\texttt{storage}}就是里面的\colc{文件}。而\colb{虚拟
  机},则像是一台(没有存储的)\colb{电脑}。而调用某个合约其实就像是在虚拟机接上
这个小U盘,然后跑一跑里面的程序。具体怎么跑则由用户来决定(这个被写在ABI里)见\cref{fig:vm-abi-acn-relationship}。

\begin{figure}[h]
  \begin{center}
    \begin{tikzpicture}
      % \draw[style=help lines] (-5cm,5cm) grid +(15cm,-10cm);
      \newdimen\myX
      \myX=4cm
      
      \path node[name=VM] {
        \raisebox{-.3\height}{\includesvg[width=2cm]{img/drive.svg}} 虚拟机VM
      } -- +(-\myX,-\myX)
      node[name=ABI]{
        \raisebox{-.3\height}{\includesvg[width=2cm]{img/keyboard.svg}} ABI
      } -- +(\myX,-\myX) node[name=Acn]{
        \raisebox{-.3\height}{\includesvg[width=1cm]{img/usb.svg}} 合约账户Account
      };

      \draw[-latex,thick] (ABI.east) ..controls + (4cm,0) ..
      node[pos=0.1,sloped,above] {指定}
      node[pos=0.2,sloped,above] {合约}
      node[pos=0.35,sloped,above] {调用}
      (VM);

      \draw[-latex,thick] ([xshift=0.5cm]VM.east) --
      node[pos=0.5,sloped,above] {改变内部存储}
      (Acn.north);
    \end{tikzpicture}
  \end{center}
  \par% or empty line, needed to get caption below the image, not to the rigth
  \adjustbox{minipage=12cm,center}{
    \caption{虚拟机,用户ABI以及合约账户之间关系的一个比喻:虚拟机不储存状态就像一
      个CPU一样,状态和内存都存在账户里,而用户提供的ABI会调用账户里的\cola{程序}来
      改变帐户里的\colc{文件}。}
  }
  \label{fig:vm-abi-acn-relationship}
\end{figure}

\subsection{交易 Transaction}
\label{sec:tx}
接下来我们介绍同样重要的 \cola{交易} 数据结构。外部世界通过交易来对区块链世界的状态进行改变,如\cref{fig:tx-world}。

\begin{figure}[h]
  \centering
  \begin{tikzpicture}
    % \draw[style=help lines] (-2cm,5cm) grid +(15cm,-10cm);
    \node[name=globe]{
      \raisebox{-.3\height}{\includesvg[width=2cm]{img/globe2.svg}}
    };
    \node at ([yshift=2cm]globe){外部世界};

    \matrix (M1) [matrix of nodes,column sep=2cm,row sep=2cm] at (10cm,0cm){
      \raisebox{-.3\height}{\includesvg[width=1cm]{img/hdd-stack-fill.svg}}
      &\raisebox{-.3\height}{\includesvg[width=1cm]{img/hdd-stack-fill.svg}}
      \\
        \raisebox{-.3\height}{\includesvg[width=1cm]{img/hdd-stack-fill.svg}}
      &\raisebox{-.3\height}{\includesvg[width=1cm]{img/hdd-stack-fill.svg}}
      \\
    };

    \node at ([yshift=3cm]M1){区块链集群};

    \tikzstyle{myArrow} = [thick,latex-latex,shorten >=10pt, shorten <=10pt,draw=\mycola]
    %                                        --- shortens the start of line
    \draw[myArrow] (M1-1-1) -- (M1-1-2);
    \draw[myArrow] (M1-1-1) -- (M1-2-1);
    \draw[myArrow] (M1-2-1) -- (M1-2-2);
    \draw[myArrow] (M1-1-2) -- (M1-2-2);

    % The Tx ->
    \draw[-latex,thick,shorten >=10pt, shorten <=10pt,color=gray] (globe) -- (M1) node[midway,above,yshift=2pt]{
      发起\cola{交易}改变区块链\colb{状态}。
    }
    node[midway,above,yshift=0.7cm]{
      \raisebox{-.3\height}{\includesvg[width=0.9cm]{img/envelope.svg}}
    }
    ;
    
  \end{tikzpicture}
  \caption{外部世界通过交易对区块链世界状态进行改变}
  \label{fig:tx-world}
\end{figure}

每条交易包括:
\begin{itemize}
\item 发起方: \texttt{from}
\item 接收方: \texttt{to}
\item 交易数据: \texttt{data}
\item 交易序列号(UUID) :\texttt{nonce}
\item 交易进入区块链集群的时间戳: \texttt{timestamp}
\item 交易哈希: \texttt{hash}
\end{itemize}

具体实现见\cref{code:tx}。

\begin{numberedc}{交易}{\label{code:tx}}
 class Tx: virtual public IJsonizable
        , virtual public ISerializable{
public:
  Tx() = default; // Upon you make a constructor yourself, the build-in
  // constructor is gone,use this to bring it back

  Tx(const address f,const address t,const bytes d,const uint64_t n):
    from(f),to(t),data(d),nonce(n){
    timestamp = std::time(nullptr);
    // convert nonce to array
    // ...计算交易hash
  };

  address from;
  /// if to is empty, create contract, else call
  address  to;
  /// if CREATE: contract bin code, else abi input
  bytes data;
  /// unique_id
  uint64_t nonce;
  /// ms since Epoch
  std::time_t timestamp;
  // usually = std::time(nullptr);
  /// Tx hash
  hash256 hash;
  // 其他方法略...
}
\end{numberedc}

\section{上层数据结构}
\label{sec:upper-data-sec}

这个章节我们介绍上层数据结构,这些结构(类)会操作并处理下层的数据结构如\cola{账
  户Account}以及\colb{交易Transaction}.

\subsection{交易执行器接口 \texttt{ITxExecutable}}
\label{sec:ITxExecutable}

如果执行模块只暴露一个接口,那么将会是什么呢? 没错,那必然就是\cola{执行交
  易\texttt{executeTx()}}。那么既然是执行交易,就一定会访问下面的\cola{世界存
  储}。更准确的说是一个\Cola{只可以读的状态存储(readonly state storage)}。那么再结合之前
\cref{sec:general-world-storage}里提到过的:
\begin{center}
  \cola{世界状态是\textbf{地址address}到\textbf{账户account}的KV表}
\end{center}
我们可以得出世界状态的接口,见\cref{code:IAcnGettable}
\begin{numberedc}{只读的世界状态接口\texttt{IAcnGettable} (这个接口是专门给交易执行的)}{\label{code:IAcnGettable}}
  class IAcnGettable{
  public:
    virtual optional<Acn> getAcn(evmc::address addr) const noexcept=0;
  };\end{numberedc}
举个例子,这个接口可以这么使用:
\begin{simplec}
Tx t = ...;                     // 交易
IAcnGettable* w = ...;          // 执行器
evmc::address a = evmc::address(static_cast<uint64_t>(123));
optional<Acn> r = w->getAcn(a);          // 获取address为123的Account
if (r):                                  // 读到了
  Acn acn = r.value();                   // 获得Account
\end{simplec}
那么这样一来我们就可以定义\cola{可执行交易}的接口了,见\cref{code:ITxExecutable}。:
\begin{numberedc}{可执行交易的接口\texttt{ITxExecutable}}{\label{code:ITxExecutable}}
class ITxExecutable {
public:
  /**
   * @brief Probably the most important function that the execution module expose.
   *
   * If the execution is successful, a series of state-changes (called
   * "journal") and the execution-result is returned.
   *
   * @param w The read-only stateDb (world state).
   * @param t The transaction to be executed.
   * @return the pair (journal,result). {} if the execution fails.
   */
  virtual optional<tuple<vector<StateChange>,bytes>> executeTx(IAcnGettable * const w,
                                                               const Tx & t) const noexcept = 0;
};
\end{numberedc}
正如\cref{code:ITxExecutable}里的注释所说,如果交易执行成功的话将会返回一个对儿:
\begin{center}
  \cola{<日志 Journal>} , \colb{<交易执行结果 Result>}
\end{center}
类型为
\begin{center}
  \cola{\texttt{vector<StateChange>}} , \colb{\texttt{bytes}}
\end{center}
其中,\texttt{StateChange} 代表一个对KV表改变的操作,见\cref{code:StateChange}:
\begin{numberedc}{状态改变\texttt{StateChange}}{\label{code:StateChange}}
  /**
   * @brief A state change, should be generated by executor.
   */
  struct StateChange{bool del=false; string k; string v;};
\end{numberedc}
一组\cola{状态改变}会组成一个\cola{日志Journal},代表着一条交易执行后对\cola{世界状态}的改变。
比如说如果你手里有一个\texttt{ITxExecutable}则可以这么用:

\begin{simplec}
IAcnGettable * w = ...;         // 世界状态
Tx t = ...;                     // 交易
ITxExecutable * e = ...;        // 执行器

auto r = e->executeTx(w,t);
if (r){
  auto [journal , result] = r.value(); // 获取结果
}
\end{simplec}

因此这两个接口的关系是这样的(\cref{fig:exe-storage-relationship}):
\begin{figure}[h]
  \centering
  \begin{tikzpicture}
    % \draw[style=help lines] (-2cm,5cm) grid +(15cm,-10cm);
    \node (ITxExecutable) [text width=8cm] at (8cm,3cm)
    {\begin{ifaceBox}[title=\texttt{ITxExecutable}]
        \begin{simplec}
  optional<tuple<vector<StateChange>,bytes>>
             executeTx(IAcnGettable *w,Tx t);
        \end{simplec}
        \end{ifaceBox}};

      \node (IAcnGettable) [text width=7cm] 
      {\begin{ifaceBox}[title=\texttt{IAcnGettable}]
          \begin{simplec}
          optional<Acn> getAcn(evmc::address addr)
\end{simplec}
          \end{ifaceBox}};
        \Southtoeast{[xshift=-3cm]ITxExecutable}{IAcnGettable}
  \end{tikzpicture}
  \caption{执行模块接口和存储模块接口的依赖关系}
  \label{fig:exe-storage-relationship}
\end{figure}
\FloatBarrier                   % \usepackage{placeins}
\subsection{EVM交易执行器 \texttt{EvmExecutor}, 一个\texttt{ITxExecutable}的实现}
\label{sec:EvmExecutor}
\texttt{EvmExecutor} 是一个 \texttt{ITxExecutable} 的实现,用来执行EVM交易。这个
类里面包了一个EVM,不存储任何状态(stateless),因此一般这么使用:

\begin{simplec}
IAcnGettable *w = ...;
unique_ptr<ITxExecutable> e{new EvmExecutor()};
Tx t = ...;                     // 交易
auto o = e->executeTx(w,t);     // 执行EVM交易
if (o)
  auto [journal,res] = o.value(); // 获取值
\end{simplec}
因此我们就有了如下关系(\cref{fig:exe-storage-relationship2})
\begin{figure}[h]
  \centering
  \begin{tikzpicture}
    % \draw[style=help lines] (-2cm,5cm) grid +(15cm,-10cm);
    \node (ITxExecutable) [text width=8cm] at (8cm,3cm)
    {\begin{ifaceBox}[title=\texttt{ITxExecutable}]
        \begin{simplec}
  optional<tuple<vector<StateChange>,bytes>>
             executeTx(IAcnGettable *w,Tx t);
        \end{simplec}
      \end{ifaceBox}};
    \node (IAcnGettable) [text width=7cm] 
    {\begin{ifaceBox}[title=\texttt{IAcnGettable}]
        \begin{simplec}
          optional<Acn> getAcn(evmc::address addr)
\end{simplec}
\end{ifaceBox}};

\node (EvmExecutor) [text width=8cm] at (8cm,-3cm)
{\begin{varBox}[title=\texttt{EvmExecutor}]
     一个EVM交易执行器的实现
    \end{varBox}};

\Southtoeast{[xshift=-3cm]ITxExecutable}{IAcnGettable}
\draw[very thick, -latex] (EvmExecutor) -- (ITxExecutable);
\end{tikzpicture}
\caption{执行模块接口和存储模块接口,EVM执行器的依赖关系}
\label{fig:exe-storage-relationship2}
\end{figure}


\subsection{可读写世界状态存储的接口 \texttt{IWorldChainStateSettable}}
\label{sec:IWorldChainStateSettable}

在每笔\cola{交易}被执行过后,我们可以获得改变世界状态
的\cola{日志Journal}和\colb{结果Result},这些终究需要被写进\colc{世界存储World
  Storage}之中。而这就需要存储暴露一个\textbf{可以读写}的接口。

回忆我们在\cref{sec:general-world-storage}里面就有说过,区块链数据是\cola{账本}
+ \colb{状态},而背后对应的则是两个DB:
\begin{enumerate}
\item \cola{\texttt{chainDB}} :  对应 \cola{账本},这个DB只能添加,不能删除。
\item \colb{\texttt{stateDB}} :  对应 \colb{状态},这个可以被日志journal所改变。
\end{enumerate}

因此我们要求的接口自然就变成了:

\begin{enumerate}
\item \texttt{get/set} \texttt{chainDB}: 往\cola{\texttt{chainDB}}里“存”和“读”的
  基本的KV操作。
\item \texttt{applyJournal}: 往\colb{\texttt{stateDB}}里用一系列日志Journal。(把
  日志里对应的KV该删的删,该加的加)。
\end{enumerate}

于是这里我们可以来介绍我们的\cola{可读写世界状态存储的接
  口 \texttt{IWorldChainStateSettable}} 了。(\cref{code:IWorldChainStateSettable})

\begin{numberedc}{可读写世界状态存储的接口\texttt{IWorldChainStateSettable}}
{\label{code:IWorldChainStateSettable}}
  class IWorldChainStateSettable {
  public:
    virtual bool setInChainDB(const string k, const string v) =0; // 往chainDB里存
    virtual optional<string> getFromChainDB(const string k) const =0;// 从chainDB里读
    virtual bool applyJournalStateDB(const vector<StateChange> & j)=0;// 对stateDB apply日志
  };
\end{numberedc}

比如说,如果想要对\texttt{stateDB}进行改变的话一般是这样的:
\begin{simplec}
IWorldChainStateSettable * p = ...;

vector<StateChange> j = {
  {false, "k1", "v1"},          // 添加(k1,v1)
  {false, "k2", "v2"},          // 添加(k2,v2)
  {true, "k3", ""},             // 删(k3)
};

p->applyJournalStateDB(j);      // apply
\end{simplec}

而我们也看到了,\texttt{chainDB}就是个\texttt{string-string}的KV,因此可以想存什么就存什么,比如:
\begin{simplec}
IWorldChainStateSettable* p = ...;

bool ok = p->setInChainDB("/blockNumber","123"); // 存入
auto r = p->getFromChainDB("/blockNumber");
if(r)                           // 读到了
  string v = r.value(); // 读出
\end{simplec}

\subsection{世界存储 \texttt{WorldStorage} 实例}
\label{sec:world-storage}

终于,我们这里介绍真正的世界存储:\texttt{WorldStorage},一个:
\begin{itemize}
\item 符合以前要求 的\texttt{IAcnGettable}(\cref{sec:ITxExecutable})以及
\item \texttt{IWorldChainStateSettable} (\cref{sec:IWorldChainStateSettable}) 接口要求
\end{itemize}
的实例: 一个真正的(暂时)\cola{区块链存储World Storage}: \texttt{WorldStorage}
(\cref{code:WorldStorage})

\begin{numberedc}{世界存储}{\label{code:WorldStorage}}
class WorldStorage: public virtual IWorldChainStateSettable,
                      public virtual IAcnGettable
  {
  public:
    rocksdb::DB* chainDB;
    rocksdb::DB* stateDB;
  // ...
}
\end{numberedc}

可以看到这个类底下的实现是两个RocksDB, 而构建函数是会接受一个文件夹路经,之后在这
个路径里会生成两个文件夹\texttt{chainDB}还有\texttt{stateDB},并分别储存这两
个RocksDB的数据。
\begin{simplec}
  WorldStorage w{};               // 在当前目录存储

  // #include<filesystem>
  namespace filesystem =  std::filesystem;
  WorldStorage w{filesystem::temp_directory_path()}; // 在tmp目录存储

  IAcnGettable* p1 = dynamic_cast<IAcnGettable*>(&w); // 给执行接口用
  IWorldChainStateSettable* p2 = dynamic_cast<IWorldChainStateSettable*>(&w); // 可读写接口
\end{simplec}
因此最后我们执行存储模块的关系图就变成了如\cref{fig:exe-storage-relationship3}:

\begin{landscape}
 \begin{figure}[h]
  \centering
  \begin{tikzpicture}
    % \draw[style=help lines] (-2cm,5cm) grid +(15cm,-10cm);
    \node (ITxExecutable) [text width=8cm] at (8cm,3cm)
    {\begin{ifaceBox}[title=\texttt{ITxExecutable}]
        \begin{simplec}
          optional<tuple<vector<StateChange>,bytes>>
          executeTx(IAcnGettable *w,Tx t);
        \end{simplec}
      \end{ifaceBox}};
    \node (IAcnGettable) [text width=7cm] 
    {\begin{ifaceBox}[title=\texttt{IAcnGettable}]
        \begin{simplec}
          optional<Acn> getAcn(evmc::address addr)
        \end{simplec}
      \end{ifaceBox}};

    
    \node (IWorldChainStateSettable) [text width=8.5cm]  at (-3cm,8cm)
    {\begin{ifaceBox}[title=\texttt{IWorldChainStateSettable}]
        \begin{simplec}
          bool setInChainDB(string k,string v)
          optional<string> getFromChainDB(string k)
          bool applyJournalStateDB(vector<StateChange> j)
        \end{simplec}
      \end{ifaceBox}};

    \node (EvmExecutor) [text width=8cm] at (8cm,-3cm)
    {\begin{varBox}[title=\texttt{EvmExecutor}]
        一个EVM交易执行器的实现
      \end{varBox}};

    \node (WorldStorage) [text width=8cm] at (-8cm,4cm)
    {\begin{varBox}[title=\texttt{WorldStorage}]
        基于RocksDB的存储
      \end{varBox}};

    \Southtoeast{[xshift=-3cm]ITxExecutable}{IAcnGettable}
    % \easttosouth[o-]{IAcnGettable}{[xshift=-3cm]ITxExecutable}

    \Southtowest[-latex]{[xshift=3cm]WorldStorage}{IAcnGettable}
    \Northtowest[-latex]{WorldStorage}{IWorldChainStateSettable}
    \draw[very thick, -latex] (EvmExecutor) -- (ITxExecutable);
  \end{tikzpicture}
  \caption{执行模块和存储模块对内和对外的接口。上层模块一般调
    用\texttt{IWorldChainStateSettable}以及\texttt{ITxExecutable}}
  \label{fig:exe-storage-relationship3}
\end{figure}
\end{landscape}


\section{区块层数据结构}
\label{sec:Block-Exec}

\cref{sec:upper-data-sec}里描述并最后提供的接口是一个很独立,且设计就是要可以被很
容易地接入其他系统里的。这里我们就介绍一个基于上面所说的一个区块执行系统。

\subsection{区块 Block}
\label{sec:Blk}


所谓区块链肯定是要有\cola{区块Block},每个区块包含如下信息:
\begin{itemize}
\item 区块号 \texttt{number}
\item 区块哈希 \texttt{hash}
\item 上一个区块的哈希 \texttt{parentHash}
\item 区块里包含的交易 Transactions \texttt{txs}
\end{itemize}
因此我们有如下定义(\cref{code:Blk}):
\begin{numberedc}{区块Block 数据结构}{\label{code:Blk}}
class Blk: virtual public IJsonizable
         , virtual public ISerializable {
public:
  Blk() = default;
  Blk(const uint64_t n,const hash256 p,vector<Tx> t):
    number(n),
    txs(t),
    parentHash(p){/*构建区块,计算哈希*/};
  uint64_t number;
  hash256 hash;
  hash256 parentHash; /// empty for genesis block
  vector<Tx> txs;
  /*其他方法略了*/
}
\end{numberedc}

\subsection{已执行区块 Executed Block}
\label{sec:ExecBlk}

每个区块在被执行过后会新获得这些信息:
\begin{itemize}
\item 每条区块内交易所导致的\colb{状态改变State Changes}。
\item 每条区块内交易所对应的\cola{交易回执Transaction Receipts}。
\end{itemize}

这些都和\cref{sec:ITxExecutable}里所描述的执行结果对应。
具体的数据定义为:
\begin{numberedc}{已执行区块  Executed Block}{\label{code:ExecBlk}}
class ExecBlk: virtual public IJsonizable
             , virtual public ISerializable
             , public Blk{
public:
  vector<vector<StateChange>> stateChanges;
  vector<TxReceipt> txReceipts;
  ExecBlk() = default;
  ExecBlk(Blk b,
          vector<vector<StateChange>> j,
          vector<TxReceipt> r): Blk(b)
                              ,stateChanges(j)
                              ,txReceipts(r){
    // ...
  }
};
\end{numberedc}

所以区块和已执行的区块大概关系见\cref{fig:execBlk}。
\begin{figure}[h]
  \centering
  \begin{tikzpicture}
    % \draw[style=help lines] (-2cm,5cm) grid +(15cm,-10cm);

    \tikzstyle{sField} = [fill=\mycola!70,draw,ellipse,text=white]
    \tikzstyle{sTx} = [draw,rectangle,text width=2cm,text centered,rounded corners]

    \begin{scope}
      \matrix [minimum size=2em,row sep=0.5cm,column sep=1cm,fill=\mycola!20,
      inner sep=1cm,name=execBlk,rounded corners]{
        \node[sTx,name=t1] {交易1}; & \node[sField,name=r1] {结果1};\\
        & \node[sField,name=j1] {日志1};\\
        \node[sTx,name=t2] {交易2}; & \node[sField,name=r2] {结果2};\\
        & \node[sField,name=j2] {日志2};\\
        \node[sTx,name=t3] {交易3}; & \node[sField,name=r3] {结果3};\\
        & \node[sField,name=j3] {日志3};\\
      };

      \foreach \i in {1,2,3}{
        \easttowest[-o,thick,gray]{t\i}{j\i}
        \easttowest[-o,thick,gray]{t\i}{r\i}
      }
      \node[above right,yshift=2pt] at (execBlk.north west) {\textbf{已执行区块} \texttt{ExecBlk}};
      \node[above right,yshift=2pt] at (execBlk.south west) {\texttt{<number>,<hash>,<parentHash>}};
    \end{scope}

    \begin{scope}[xshift=8cm]
      \matrix [minimum size=2em,row sep=1cm,column sep=1.5cm,fill=\mycola!15,
      inner sep=1cm,name=Blk,rounded corners]{
        \node[sTx,name=t1] {交易1}; & \\ 
        \node[sTx,name=t2] {交易2}; & \\
        \node[sTx,name=t3] {交易3}; & \\
      };

      \node[above right,yshift=2pt] at (Blk.north west) {\textbf{区块} \texttt{Blk}};
      \node[above right,yshift=2pt] at (Blk.south west) {\texttt{<number>,<hash>,<parentHash>}};
    \end{scope}
    
  \end{tikzpicture}
  \caption{区块和已执行区块}
\label{fig:execBlk}
\end{figure}

\section{区块执行接口}
\label{sec:IBlkExecutable}

就像和交易执行接口一样(\cref{sec:ITxExecutable})这里我们同样有区块执行接口
\begin{numberedc}{IBlkExecutable}{\label{code:IBlkExecutable}}
  class IBlkExecutable {
  public:
    virtual ExecBlk executeBlk(const Blk & b) const noexcept = 0;
    virtual bool commitBlk(const ExecBlk & b) noexcept = 0;
  };
\end{numberedc}
其中\texttt{executeBlk()}会执行一个区块,第二个(语义上)会把执行过的区块写入存储。
而真正符合这个接口的实例则是\texttt{BlkExecutor}(\cref{code:BlkExecutor}):

\begin{numberedc}{BlkExecutor}{\label{code:BlkExecutor}}
class BlkExecutor: public virtual IBlkExecutable{
public:
  IWorldChainStateSettable* const world;
  ITxExecutable* const txExecutor;
  IAcnGettable* const readOnlyWorld;
  BlkExecutor(IWorldChainStateSettable* const w,
              ITxExecutable* const e,
              IAcnGettable* const r
              ): world(w),
                 txExecutor(e),
                 readOnlyWorld(r){
    // ...
  };
  // 具体实现...
}
\end{numberedc}

可以看到这个\texttt{BlkExecutor}会接受之前在\cref{sec:upper-data-sec}之中定义的三个接口。(见\ref{fig:exe-storage-relationship4})

\begin{landscape}
 \begin{figure}[h]
  \centering
  \begin{tikzpicture}
    % \draw[style=help lines] (-2cm,5cm) grid +(15cm,-10cm);
    \node (ITxExecutable) [text width=8cm] at (8cm,3cm)
    {\begin{ifaceBox}[title=\texttt{ITxExecutable}]
        \begin{simplec}
          optional<tuple<vector<StateChange>,bytes>>
          executeTx(IAcnGettable *w,Tx t);
        \end{simplec}
      \end{ifaceBox}};
    \node (IAcnGettable) [text width=7cm] 
    {\begin{ifaceBox}[title=\texttt{IAcnGettable}]
        \begin{simplec}
          optional<Acn> getAcn(evmc::address addr)
        \end{simplec}
      \end{ifaceBox}};

    
    \node (IWorldChainStateSettable) [text width=8.5cm]  at (-3cm,8cm)
    {\begin{ifaceBox}[title=\texttt{IWorldChainStateSettable}]
        \begin{simplec}
          bool setInChainDB(string k,string v)
          optional<string> getFromChainDB(string k)
          bool applyJournalStateDB(vector<StateChange> j)
        \end{simplec}
      \end{ifaceBox}};

    \node (EvmExecutor) [text width=8cm] at (8cm,-3cm)
    {\begin{varBox}[title=\texttt{EvmExecutor}]
        一个EVM交易执行器的实现
      \end{varBox}};

    \node (WorldStorage) [text width=8cm] at (-8cm,4cm)
    {\begin{varBox}[title=\texttt{WorldStorage}]
        基于RocksDB的存储
      \end{varBox}};

    \node (BlkExecutor) [text width=8cm] at (8cm,6cm)
    {\begin{varBox}[title=\texttt{BlkExecutor}]
        区块执行器
      \end{varBox}};

    \node (IBlkExecutable) [text width=8.5cm]  at (8cm,9cm)
    {\begin{ifaceBox}[title=\texttt{IBlkExecutable}]
        \begin{simplec}
    ExecBlk executeBlk(Blk & b)
    bool commitBlk(ExecBlk & b)\end{simplec}
      \end{ifaceBox}};

    \easttowest[o-]{IWorldChainStateSettable}{BlkExecutor}
    \southtonorth[latex-]{IBlkExecutable}{BlkExecutor}
    \westTonorth[-o]{BlkExecutor}{IAcnGettable}
    % \Northtowest[o-]{IAcnGettable}{BlkExecutor}
    \westTowest[-o]{BlkExecutor}{ITxExecutable}

    \Southtoeast{[xshift=-3cm]ITxExecutable}{IAcnGettable}
    % \easttosouth[o-]{IAcnGettable}{[xshift=-3cm]ITxExecutable}
    \Southtowest[-latex]{[xshift=3cm]WorldStorage}{IAcnGettable}

    \Northtowest[-latex]{WorldStorage}{IWorldChainStateSettable}
    \draw[very thick, -latex] (EvmExecutor) -- (ITxExecutable);
  \end{tikzpicture}
  \caption{区块执行器\texttt{IBlkExecutable}所使用和调用的接口}
  \label{fig:exe-storage-relationship4}
\end{figure}
\end{landscape}
% Local Variables:
% TeX-engine: luatex
% TeX-command-extra-options: "-shell-escape"
% TeX-master: "m.tex"
% End:

\section{基准测试 Benchmark}
\label{sec:benchmark}

\subsection{测试结果}

执行和存储模块的基准测试结果见\cref{fig:benchmark}。每个模块分别进行了5次基准测试。
在给定的执行环境下 (\cref{sec:benchmark-detail}),所有模块的执行TPS均符合标准。
\begin{figure}[h]
  \centering
  \caption{执行存储模块基准测试}
  % Created by tikzDevice version 0.12.4 on 2023-07-05 18:31:39
% !TEX encoding = UTF-8 Unicode
\begin{tikzpicture}[x=1pt,y=1pt]
\definecolor{fillColor}{RGB}{255,255,255}
\path[use as bounding box,fill=fillColor,fill opacity=0.00] (0,0) rectangle (505.89,289.08);
\begin{scope}
\path[clip] (  0.00,  0.00) rectangle (505.89,289.08);
\definecolor{drawColor}{RGB}{255,255,255}
\definecolor{fillColor}{RGB}{255,255,255}

\path[draw=drawColor,line width= 0.6pt,line join=round,line cap=round,fill=fillColor] (  0.00,  0.00) rectangle (505.89,289.08);
\end{scope}
\begin{scope}
\path[clip] ( 78.46, 31.63) rectangle (393.51,283.58);
\definecolor{fillColor}{gray}{0.92}

\path[fill=fillColor] ( 78.46, 31.63) rectangle (393.51,283.58);
\definecolor{drawColor}{RGB}{255,255,255}

\path[draw=drawColor,line width= 0.3pt,line join=round] (129.81, 31.63) --
	(129.81,283.58);

\path[draw=drawColor,line width= 0.3pt,line join=round] (203.85, 31.63) --
	(203.85,283.58);

\path[draw=drawColor,line width= 0.3pt,line join=round] (277.89, 31.63) --
	(277.89,283.58);

\path[draw=drawColor,line width= 0.3pt,line join=round] (351.94, 31.63) --
	(351.94,283.58);

\path[draw=drawColor,line width= 0.6pt,line join=round] ( 78.46, 78.87) --
	(393.51, 78.87);

\path[draw=drawColor,line width= 0.6pt,line join=round] ( 78.46,157.61) --
	(393.51,157.61);

\path[draw=drawColor,line width= 0.6pt,line join=round] ( 78.46,236.34) --
	(393.51,236.34);

\path[draw=drawColor,line width= 0.6pt,line join=round] ( 92.78, 31.63) --
	( 92.78,283.58);

\path[draw=drawColor,line width= 0.6pt,line join=round] (166.83, 31.63) --
	(166.83,283.58);

\path[draw=drawColor,line width= 0.6pt,line join=round] (240.87, 31.63) --
	(240.87,283.58);

\path[draw=drawColor,line width= 0.6pt,line join=round] (314.92, 31.63) --
	(314.92,283.58);

\path[draw=drawColor,line width= 0.6pt,line join=round] (388.96, 31.63) --
	(388.96,283.58);
\definecolor{fillColor}{RGB}{255,108,145}

\path[fill=fillColor] ( 92.78, 59.19) rectangle (342.13, 67.06);
\definecolor{fillColor}{RGB}{227,128,150}

\path[fill=fillColor] ( 92.78, 67.06) rectangle (336.33, 74.94);
\definecolor{fillColor}{RGB}{199,143,153}

\path[fill=fillColor] ( 92.78, 74.94) rectangle (365.91, 82.81);
\definecolor{fillColor}{RGB}{174,153,156}

\path[fill=fillColor] ( 92.78, 82.81) rectangle (270.05, 90.68);
\definecolor{fillColor}{gray}{0.62}

\path[fill=fillColor] ( 92.78, 90.68) rectangle (357.34, 98.56);
\definecolor{fillColor}{RGB}{255,108,145}

\path[fill=fillColor] ( 92.78,137.92) rectangle (311.74,145.80);
\definecolor{fillColor}{RGB}{227,128,150}

\path[fill=fillColor] ( 92.78,145.80) rectangle (319.99,153.67);
\definecolor{fillColor}{RGB}{199,143,153}

\path[fill=fillColor] ( 92.78,153.67) rectangle (306.52,161.54);
\definecolor{fillColor}{RGB}{174,153,156}

\path[fill=fillColor] ( 92.78,161.54) rectangle (379.19,169.42);
\definecolor{fillColor}{gray}{0.62}

\path[fill=fillColor] ( 92.78,169.42) rectangle (319.07,177.29);
\definecolor{fillColor}{RGB}{255,108,145}

\path[fill=fillColor] ( 92.78,216.66) rectangle (296.07,224.53);
\definecolor{fillColor}{RGB}{227,128,150}

\path[fill=fillColor] ( 92.78,224.53) rectangle (291.00,232.40);
\definecolor{fillColor}{RGB}{199,143,153}

\path[fill=fillColor] ( 92.78,232.40) rectangle (277.80,240.28);
\definecolor{fillColor}{RGB}{174,153,156}

\path[fill=fillColor] ( 92.78,240.28) rectangle (289.01,248.15);
\definecolor{fillColor}{gray}{0.62}

\path[fill=fillColor] ( 92.78,248.15) rectangle (292.28,256.02);
\definecolor{drawColor}{RGB}{0,0,0}

\path[draw=drawColor,line width= 0.6pt,line join=round] (265.55, 31.63) -- (265.55,283.58);

\node[text=drawColor,anchor=base west,inner sep=0pt, outer sep=0pt, scale=  1.14] at (274.20,109.98) {7万TPS基准线};
\end{scope}
\begin{scope}
\path[clip] (  0.00,  0.00) rectangle (505.89,289.08);
\definecolor{drawColor}{gray}{0.30}

\node[text=drawColor,anchor=base east,inner sep=0pt, outer sep=0pt, scale=  0.88] at ( 73.51, 75.68) {存储模块 (存入)};

\node[text=drawColor,anchor=base east,inner sep=0pt, outer sep=0pt, scale=  0.88] at ( 73.51,154.41) {存储模块 (读取)};

\node[text=drawColor,anchor=base east,inner sep=0pt, outer sep=0pt, scale=  0.88] at ( 73.51,233.15) {执行模块};
\end{scope}
\begin{scope}
\path[clip] (  0.00,  0.00) rectangle (505.89,289.08);
\definecolor{drawColor}{gray}{0.20}

\path[draw=drawColor,line width= 0.6pt,line join=round] ( 75.71, 78.87) --
	( 78.46, 78.87);

\path[draw=drawColor,line width= 0.6pt,line join=round] ( 75.71,157.61) --
	( 78.46,157.61);

\path[draw=drawColor,line width= 0.6pt,line join=round] ( 75.71,236.34) --
	( 78.46,236.34);
\end{scope}
\begin{scope}
\path[clip] (  0.00,  0.00) rectangle (505.89,289.08);
\definecolor{drawColor}{gray}{0.20}

\path[draw=drawColor,line width= 0.6pt,line join=round] ( 92.78, 28.88) --
	( 92.78, 31.63);

\path[draw=drawColor,line width= 0.6pt,line join=round] (166.83, 28.88) --
	(166.83, 31.63);

\path[draw=drawColor,line width= 0.6pt,line join=round] (240.87, 28.88) --
	(240.87, 31.63);

\path[draw=drawColor,line width= 0.6pt,line join=round] (314.92, 28.88) --
	(314.92, 31.63);

\path[draw=drawColor,line width= 0.6pt,line join=round] (388.96, 28.88) --
	(388.96, 31.63);
\end{scope}
\begin{scope}
\path[clip] (  0.00,  0.00) rectangle (505.89,289.08);
\definecolor{drawColor}{gray}{0.30}

\node[text=drawColor,anchor=base,inner sep=0pt, outer sep=0pt, scale=  0.88] at ( 92.78, 20.29) {0};

\node[text=drawColor,anchor=base,inner sep=0pt, outer sep=0pt, scale=  0.88] at (166.83, 20.29) {3};

\node[text=drawColor,anchor=base,inner sep=0pt, outer sep=0pt, scale=  0.88] at (240.87, 20.29) {6};

\node[text=drawColor,anchor=base,inner sep=0pt, outer sep=0pt, scale=  0.88] at (314.92, 20.29) {9};

\node[text=drawColor,anchor=base,inner sep=0pt, outer sep=0pt, scale=  0.88] at (388.96, 20.29) {12};
\end{scope}
\begin{scope}
\path[clip] (  0.00,  0.00) rectangle (505.89,289.08);
\definecolor{drawColor}{RGB}{0,0,0}

\node[text=drawColor,anchor=base,inner sep=0pt, outer sep=0pt, scale=  1.10] at (235.99,  7.75) {Transaction Per Seconds TPS(万)};
\end{scope}
\begin{scope}
\path[clip] (  0.00,  0.00) rectangle (505.89,289.08);
\definecolor{fillColor}{RGB}{255,255,255}

\path[fill=fillColor] (404.51,108.10) rectangle (500.39,207.11);
\end{scope}
\begin{scope}
\path[clip] (  0.00,  0.00) rectangle (505.89,289.08);
\definecolor{drawColor}{RGB}{0,0,0}

\node[text=drawColor,anchor=base west,inner sep=0pt, outer sep=0pt, scale=  1.10] at (410.01,192.50) {基准测试};
\end{scope}
\begin{scope}
\path[clip] (  0.00,  0.00) rectangle (505.89,289.08);
\definecolor{fillColor}{gray}{0.95}

\path[fill=fillColor] (410.01,171.42) rectangle (424.47,185.87);
\end{scope}
\begin{scope}
\path[clip] (  0.00,  0.00) rectangle (505.89,289.08);
\definecolor{fillColor}{RGB}{255,108,145}

\path[fill=fillColor] (410.72,172.13) rectangle (423.76,185.16);
\end{scope}
\begin{scope}
\path[clip] (  0.00,  0.00) rectangle (505.89,289.08);
\definecolor{fillColor}{gray}{0.95}

\path[fill=fillColor] (410.01,156.96) rectangle (424.47,171.42);
\end{scope}
\begin{scope}
\path[clip] (  0.00,  0.00) rectangle (505.89,289.08);
\definecolor{fillColor}{RGB}{227,128,150}

\path[fill=fillColor] (410.72,157.67) rectangle (423.76,170.71);
\end{scope}
\begin{scope}
\path[clip] (  0.00,  0.00) rectangle (505.89,289.08);
\definecolor{fillColor}{gray}{0.95}

\path[fill=fillColor] (410.01,142.51) rectangle (424.47,156.96);
\end{scope}
\begin{scope}
\path[clip] (  0.00,  0.00) rectangle (505.89,289.08);
\definecolor{fillColor}{RGB}{199,143,153}

\path[fill=fillColor] (410.72,143.22) rectangle (423.76,156.25);
\end{scope}
\begin{scope}
\path[clip] (  0.00,  0.00) rectangle (505.89,289.08);
\definecolor{fillColor}{gray}{0.95}

\path[fill=fillColor] (410.01,128.06) rectangle (424.47,142.51);
\end{scope}
\begin{scope}
\path[clip] (  0.00,  0.00) rectangle (505.89,289.08);
\definecolor{fillColor}{RGB}{174,153,156}

\path[fill=fillColor] (410.72,128.77) rectangle (423.76,141.80);
\end{scope}
\begin{scope}
\path[clip] (  0.00,  0.00) rectangle (505.89,289.08);
\definecolor{fillColor}{gray}{0.95}

\path[fill=fillColor] (410.01,113.60) rectangle (424.47,128.06);
\end{scope}
\begin{scope}
\path[clip] (  0.00,  0.00) rectangle (505.89,289.08);
\definecolor{fillColor}{gray}{0.62}

\path[fill=fillColor] (410.72,114.31) rectangle (423.76,127.34);
\end{scope}
\begin{scope}
\path[clip] (  0.00,  0.00) rectangle (505.89,289.08);
\definecolor{drawColor}{RGB}{0,0,0}

\node[text=drawColor,anchor=base west,inner sep=0pt, outer sep=0pt, scale=  0.88] at (429.97,175.45) {第1批基准测试};
\end{scope}
\begin{scope}
\path[clip] (  0.00,  0.00) rectangle (505.89,289.08);
\definecolor{drawColor}{RGB}{0,0,0}

\node[text=drawColor,anchor=base west,inner sep=0pt, outer sep=0pt, scale=  0.88] at (429.97,161.00) {第2批基准测试};
\end{scope}
\begin{scope}
\path[clip] (  0.00,  0.00) rectangle (505.89,289.08);
\definecolor{drawColor}{RGB}{0,0,0}

\node[text=drawColor,anchor=base west,inner sep=0pt, outer sep=0pt, scale=  0.88] at (429.97,146.54) {第3批基准测试};
\end{scope}
\begin{scope}
\path[clip] (  0.00,  0.00) rectangle (505.89,289.08);
\definecolor{drawColor}{RGB}{0,0,0}

\node[text=drawColor,anchor=base west,inner sep=0pt, outer sep=0pt, scale=  0.88] at (429.97,132.09) {第4批基准测试};
\end{scope}
\begin{scope}
\path[clip] (  0.00,  0.00) rectangle (505.89,289.08);
\definecolor{drawColor}{RGB}{0,0,0}

\node[text=drawColor,anchor=base west,inner sep=0pt, outer sep=0pt, scale=  0.88] at (429.97,117.63) {第5批基准测试};
\end{scope}
\end{tikzpicture}

  \label{fig:benchmark}
\end{figure}


具体的数值见\cref{tab:benchmark}。
\begin{table}[h]
  \centering
  \newcolumntype{b}{X}
  \newcolumntype{s}{>{\hsize=.5\hsize}X}
  \newcommand{\heading}[1]{\multicolumn{1}{c}{#1}}
  \begin{tabularx}{0.9\textwidth} { 
      bsss
    }

    \hline
    & 存储模块(存入)& 存储模块(读取) & 执行模块  \\ \hline
第1批基准测试 & 10.10& 8.87& 8.24\\
第2批基准测试 & 9.87& 9.21& 8.03\\
第3批基准测试 & 11.07& 8.66& 7.50\\
第4批基准测试 & 7.18& 11.60& 7.95\\
第5批基准测试 & 10.72& 9.17& 8.08\\\hline
样本平均值 sample mean $\mu$& 9.79& 9.50& 7.96\\
样本方差 sample variance $\sigma^{2}$& 2.35& 1.43& 0.08\\\hline
  \end{tabularx}
  \caption{基准测试数值结果 (单位: 万)}
  \label{tab:benchmark}
\end{table}
\FloatBarrier
在\cref{tab:benchmark}之中,样本平均值和样本方差计算为:

\begin{align*}
  \mu  = \frac{\sum_i(X_i)}{N}, \quad
  \sigma^2 = \frac{\sum_i(X_i - \mu)^2}{N-1}, \quad
  \text{$N$为样本数,$X_i$为样本。}
\end{align*}

\subsection{测试细节}
\label{sec:benchmark-detail}
在TPS的测试过程中有很多可以配置的参数和环境,这些都会很大程度的影响所测算出来的结
果。
\FloatBarrier
首先是环境,测试机的重要系统信息见\cref{tab:sysinfo}。
\begin{table}
  \centering
  \newcolumntype{m}{>{\hsize=.7\hsize}X}
  \begin{tabularx}{0.8\linewidth}{mX}
    \toprule
    硬件模型 hardware model& \texttt{ASUSTeK COMPUTER INC. VivoBook\_ASUSLaptop X513EP\_V5050EP}\\ \hline
    内存 Memory & \texttt{16.0 GiB}\\ \hline
    处理器 Processor & \texttt{11th Gen Intel® Core™ i5-1135G7 @ 2.40GHz × 8}\\ \hline
    操作系统 OS& \texttt{Ubuntu 22.04.2 LTS 64-bit}\\ \hline
    图形处理 Graphics& \texttt{Mesa Intel® Xe Graphics (TGL GT2)}\\
    \bottomrule
  \end{tabularx}
  \caption{测试系统主要信息}
  \label{tab:sysinfo}
\end{table}
\FloatBarrier

除环境以外还有的就是一些模块内部参数的配置,这里我们只列出来两个比较广义的(每个模块都有的)重要参数:
\begin{itemize}
\item \textbf{并发线程数}: 并发执行某个操作的线程数量。
\item \textbf{批量交易数}: 某些操作接受批量交易处理(batch processing), 如批量交
  易存入/读出。这个代表每个批量内含的交易数。
\end{itemize}
这些参数的值和结果不是一个线性关系。比如说,如果开一万个线程同时执行交易,经实
测,TPS大概为两万左右。因此,这些参数的最优值(在每个系统上不一样)都需要摸索一
番(trail-and-error) 然后找到一个中间的,达标的值,具体得到\cref{tab:benchmark}结果的参数见\cref{tab:paraminfo}。
\begin{table}[h]
  \centering
  \begin{tabularx}{0.8\linewidth}{XXX}
    \toprule
    & 并发线程数 & 批量交易数 \\
    \midrule
    执行模块        &4&50\\
    存储模块 (读取)&8&1000\\
    执行模块  (存入)&8&1000\\
    \bottomrule
  \end{tabularx}
  \caption{重要参数配置}
  \label{tab:paraminfo}
\end{table}
% \FloatBarrier
% \clearpage{}
\newpage{}

\section{附录: 接口Interface(纯虚类 Abstract Base Class)}
\label{sec:iface}
接口(Interface)和类(Class)类似,但是只指定行为(behavior)不存有状态数据。因此:
\begin{itemize}
\item 接口只定义函数(function),不定义域(field)
\item 一个类可以继承\cola{多个}接口,但是只能继承\cola{一个}类\footnote{当然
    了,C++语法确实允许你继承多个类(Multiple Inheritance),但聪明的你或许并不应
    该这么做。}
\end{itemize}
使用接口可以增加代码的模块性,这样可以更方便调试,管理,设计以及更新代码。

\subsection{ISerializable}
\label{sec:ISerializable}
这个类表示任何可以被序列化和反序列化的类:
\begin{simplec}
  class ISerializable{
    virtual bool fromString(string_view s) noexcept =0;
    virtual string toString() const noexcept =0;
  };
\end{simplec}
序列化的具体方法作为类的使用方应该是不需要知道的,可以
是JSON,XML,RLP,Protobuf..等等。


而我们看到,我们的很多类如账户(\texttt{Acn})和交易(\texttt{Tx})都符
合\texttt{ISerializable},因此可以这么使用:
\begin{simplec}
Acn a = ...;
string s = a.toString();        // Serialize
Acn a1;
BOOST_CHECK(a1.fromString(s));  // Parse
\end{simplec}
\subsection{IJsonizable}
\label{sec:IJsonizable}

这个类表示任何可以被序列化和反序列化成JSON的类:
\begin{simplec}
  class IJsonizable{
  public:
    virtual json::value toJson() const noexcept=0;
    virtual bool fromJson(const json::value &)noexcept=0;

    bool fromJsonString(string_view s) noexcept{...};
    string toJsonString() const noexcept{...};
  };
\end{simplec}
如果一个类实现了自己的\texttt{toJson()}和\texttt{fromJson()}方法的话,那这个接口
就会免费赠送\texttt{fromJsonString()}和\texttt{toJsonString()}两个方法。

暂时使用的Json库为\texttt{boost/json}(要求\texttt{Boost 1.75>=})。

和\texttt{ISerializable}一样,我们的很多类如账户(\texttt{Acn})和交易(\texttt{Tx})都符合\texttt{IJsonizable},因此可以这么使用:
\begin{simplec}
Acn a = ...;
json::value v = a.toJson();     // serialize
Acn a1,a2;
BOOST_CHECK(a1.fromJson(v));    // parse

string s = a.toJsonString();
BOOST_CHECK(a2.fromJson(s));\end{simplec}

\end{document}


% Local Variables:
% TeX-engine: luatex
% TeX-command-extra-options: "-shell-escape"
% TeX-master: "m.tex"
% TeX-parse-self: t
% TeX-auto-save: t
% End: