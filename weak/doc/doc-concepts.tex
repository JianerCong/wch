
\section{Concepts}

\colz{
  \emoji{parrot} : I don't know much about blockchain, but I did hear of
  cryptocurrency like Bitcoin. Is \Wch{} something like that?
}

\emoji{turtle} : Not quite. Blockchain is like a family. It has many members.
Cryptocurrency, like Bitcoin, is one of them. The members share some common
concepts and ideas including:

\begin{enumerate}
\item Consensus mechanisms
\item Ledger as database
\item etc...
\end{enumerate}

\Wch{} belongs one of the members called \cola{private chain}. It's called
``private'', because in contrast to public chain like Cryptocurrency, the nodes
are often run in private networks, and the data is not open to the public. It
can do things such as making sure that the data is not tampered with unless a
majority of the nodes are controlled by the same person. As a result, it can be used to
store things like contracts.

Private chain has its own core concepts. Some are shared with other members, and
some are not. These differences however, make some members quite different from
each other, even though they are all called ``blockchain''. It's like penguin
and hummingbird are both ``birds'', but they are quite different.

In this section we will go through some of the important concepts of
private chain. 

\emoji{parrot} : Okay... But I don't wanna know too much details. I just wanna
get the chain running and use it.

\emoji{turtle} : If you just want to use \Wch{}, skip to \cref{sec:tut} to see
the tutorial. You can always come back here later if you find some scary terms.

\begin{weakBox}[title=Some terms]
  \emoji{turtle} : From now on, when we say ``blockchain'', we mean all the
  whole blockchain family, and when we say ``private blockchain'' or simply
  ``private chain'', we mean the member that \Wch{} belongs to.
\end{weakBox}

\subsection{Storage and Execution of Transactions}
Blockchain is a shared database.
