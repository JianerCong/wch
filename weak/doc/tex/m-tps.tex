% \documentclass[dvipsnames]{article}
\documentclass[dvipsnames]{ctexart}

\title{FollowMe 共识}
\usepackage{geometry}\geometry{
  a4paper,
  total={170mm,257mm},
  left=20mm,
  top=20mm,
}


\usepackage{svg}

\usepackage[skip=5pt plus1pt, indent=0pt]{parskip}
% Color
\newcommand{\mycola}{MidnightBlue}
\newcommand{\mycolb}{Mahogany}
\newcommand{\mycolc}{OliveGreen}

\newcommand{\cola}[1]{\textcolor{\mycola}{#1}}
\newcommand{\colb}[1]{\textcolor{\mycolb}{#1}}
\newcommand{\colc}[1]{\textcolor{\mycolc}{#1}}
\newcommand{\Cola}[1]{\textcolor{\mycola}{\textbf{#1}}}

% \let\emph\relax % there's no \RedeclareTextFontCommand
% \DeclareTextFontCommand{\emph}{\bfseries}
\renewcommand{\emph}[1]{\texbf{#1}}

\usepackage{fontspec}

\setmonofont{Cascadia}[
Path=/usr/share/fonts/truetype/Cascadia_Code/,
Scale=0.85,
Extension = .ttf,
UprightFont=*Code,              %find CascadiaCode.ttf
BoldFont=*CodePL,               %find CascadiaCodePL.ttf ...
ItalicFont=*CodeItalic,
BoldItalicFont=*CodePLItalic
]
% --------------------------------------------------
% Windows
% \setmonofont{Cascadia}[
% Path=C:/Windows/Fonts/,
% Extension = .ttf,
% UprightFont=*Mono,              %find CascadiaMono.ttf
% BoldFont=*Code,               %find CascadiaCodePL.ttf ...
% ItalicFont=*Code,
% BoldItalicFont=*Code
% ]


\usepackage{minted}
\usepackage{tcolorbox}
\tcbuselibrary{skins}
\tcbuselibrary{minted}
\usepackage{tikz}
\usetikzlibrary{shapes} % ellipse node shape
\usetikzlibrary{shapes.multipart} % for line breaks in node text
\usetikzlibrary{arrows.meta}    %-o arrow head
\usetikzlibrary{arrows}
\usetikzlibrary{matrix}


\usepackage{amsmath}
% ??? still xelatex?
% \usepackage{xeCJK}
\usepackage{emoji}
% \setemojifont{NotoColorEmoji.ttf}[Path=C:/Users/congj/repo/myFonts/]
% \setemojifont{TwitterColorEmoji-SVGinOT.ttf}[Path=C:/Users/congj/repo/myFonts/]


\newtcolorbox[auto counter]{myBox}[2][]{
  fonttitle=\bfseries,title={共识~\thetcbcounter: #2},#1
}
\newtcolorbox[]{noteBox}[1][]{
  tile,left=1mm,nobeforeafter,fontupper=\small,#1
}

\tikzstyle{myNode}=[inner sep=2pt,circle,text=white]
\date{\today}
\author{作者}

\newtcblisting{simplec}{
  listing engine=minted,
  minted language=c++,
  minted style=vs,
  minted options={fontsize=\small,autogobble,
    % framesep=1cm
  },
  tile,
  listing only,
  % bottom=0cm,
  % nobeforeafter, 
  boxsep=0mm,
  left=1mm,
  opacityback=0.5,
  colback=gray!20
}
\tcbuselibrary{breakable}
\newtcblisting{simplepy}{
  listing engine=minted,
  minted language=python,
  minted style=vs,
  minted options={fontsize=\small,autogobble,
    % framesep=1cm
  },
  tile,
  listing only,
  % bottom=0cm,
  % nobeforeafter,
  boxsep=0mm,
  left=1mm,
  opacityback=0.5,
  colback=gray!20,
  breakable
}
\newtcolorbox{blackbox}{tile,colback=black,colupper=white,nobeforeafter,halign=flush center}

\tikzstyle{myMatrix}=[matrix of nodes,below right,
nodes={above,text centered},                  %apply to all nodes
row sep=1cm,column sep=2cm]
\tikzstyle{every node}=[inner sep=0pt]

\newcommand\uptoleft[3][-o]{\draw[very thick,#1](#2.south) |- (#3.west);}
\newcommand\uptodown[3][-o]{\draw[very thick,#1](#2.south) to [out=270,in=90] (#3.north);}
\newcommand\downtoup[3][-latex]{\draw[very thick,#1](#2.north) to [out=90,in=270] (#3.south);}

\newcommand\lefttoright[3][-latex]{\draw[very thick,#1](#2.east) to[out=0,in=180] (#3.west);}
\newcommand\lefttodown[3][-latex]{\draw[very thick,#1](#2.east) to[out=0,in=90] (#3.north);}


\newtcolorbox{leftDialogBox}{
  tile, nobeforeafter, boxsep=0pt,
  % show bounding box,
  colback=\mycola!10,
  overlay={
    \begin{scope}
      % \fill[gray!10] (frame.east) circle (2pt);
      \fill[\mycola!10] (frame.east) --
      +(0,2mm) --
      +(3mm,0) --
      +(0,-2mm)
      ;
    \end{scope}
  }}


\newtcolorbox{rightDialogBox}{
  tile, nobeforeafter, boxsep=0pt,
  % show bounding box,
  colback=\mycola!10,
  overlay={
    \begin{scope}
      % \fill[gray!10] (frame.east) circle (2pt);
      \fill[\mycola!10] (frame.west) --
      +(0,2mm) --
      +(-3mm,0) --
      +(0,-2mm);
    \end{scope}
  }}

\newcommand{\mycolaa}{\mycola!20}


\begin{document}
\maketitle
% Created by tikzDevice version 0.12.4 on 2023-07-05 18:31:39
% !TEX encoding = UTF-8 Unicode
\begin{tikzpicture}[x=1pt,y=1pt]
\definecolor{fillColor}{RGB}{255,255,255}
\path[use as bounding box,fill=fillColor,fill opacity=0.00] (0,0) rectangle (505.89,289.08);
\begin{scope}
\path[clip] (  0.00,  0.00) rectangle (505.89,289.08);
\definecolor{drawColor}{RGB}{255,255,255}
\definecolor{fillColor}{RGB}{255,255,255}

\path[draw=drawColor,line width= 0.6pt,line join=round,line cap=round,fill=fillColor] (  0.00,  0.00) rectangle (505.89,289.08);
\end{scope}
\begin{scope}
\path[clip] ( 78.46, 31.63) rectangle (393.51,283.58);
\definecolor{fillColor}{gray}{0.92}

\path[fill=fillColor] ( 78.46, 31.63) rectangle (393.51,283.58);
\definecolor{drawColor}{RGB}{255,255,255}

\path[draw=drawColor,line width= 0.3pt,line join=round] (129.81, 31.63) --
	(129.81,283.58);

\path[draw=drawColor,line width= 0.3pt,line join=round] (203.85, 31.63) --
	(203.85,283.58);

\path[draw=drawColor,line width= 0.3pt,line join=round] (277.89, 31.63) --
	(277.89,283.58);

\path[draw=drawColor,line width= 0.3pt,line join=round] (351.94, 31.63) --
	(351.94,283.58);

\path[draw=drawColor,line width= 0.6pt,line join=round] ( 78.46, 78.87) --
	(393.51, 78.87);

\path[draw=drawColor,line width= 0.6pt,line join=round] ( 78.46,157.61) --
	(393.51,157.61);

\path[draw=drawColor,line width= 0.6pt,line join=round] ( 78.46,236.34) --
	(393.51,236.34);

\path[draw=drawColor,line width= 0.6pt,line join=round] ( 92.78, 31.63) --
	( 92.78,283.58);

\path[draw=drawColor,line width= 0.6pt,line join=round] (166.83, 31.63) --
	(166.83,283.58);

\path[draw=drawColor,line width= 0.6pt,line join=round] (240.87, 31.63) --
	(240.87,283.58);

\path[draw=drawColor,line width= 0.6pt,line join=round] (314.92, 31.63) --
	(314.92,283.58);

\path[draw=drawColor,line width= 0.6pt,line join=round] (388.96, 31.63) --
	(388.96,283.58);
\definecolor{fillColor}{RGB}{255,108,145}

\path[fill=fillColor] ( 92.78, 59.19) rectangle (342.13, 67.06);
\definecolor{fillColor}{RGB}{227,128,150}

\path[fill=fillColor] ( 92.78, 67.06) rectangle (336.33, 74.94);
\definecolor{fillColor}{RGB}{199,143,153}

\path[fill=fillColor] ( 92.78, 74.94) rectangle (365.91, 82.81);
\definecolor{fillColor}{RGB}{174,153,156}

\path[fill=fillColor] ( 92.78, 82.81) rectangle (270.05, 90.68);
\definecolor{fillColor}{gray}{0.62}

\path[fill=fillColor] ( 92.78, 90.68) rectangle (357.34, 98.56);
\definecolor{fillColor}{RGB}{255,108,145}

\path[fill=fillColor] ( 92.78,137.92) rectangle (311.74,145.80);
\definecolor{fillColor}{RGB}{227,128,150}

\path[fill=fillColor] ( 92.78,145.80) rectangle (319.99,153.67);
\definecolor{fillColor}{RGB}{199,143,153}

\path[fill=fillColor] ( 92.78,153.67) rectangle (306.52,161.54);
\definecolor{fillColor}{RGB}{174,153,156}

\path[fill=fillColor] ( 92.78,161.54) rectangle (379.19,169.42);
\definecolor{fillColor}{gray}{0.62}

\path[fill=fillColor] ( 92.78,169.42) rectangle (319.07,177.29);
\definecolor{fillColor}{RGB}{255,108,145}

\path[fill=fillColor] ( 92.78,216.66) rectangle (296.07,224.53);
\definecolor{fillColor}{RGB}{227,128,150}

\path[fill=fillColor] ( 92.78,224.53) rectangle (291.00,232.40);
\definecolor{fillColor}{RGB}{199,143,153}

\path[fill=fillColor] ( 92.78,232.40) rectangle (277.80,240.28);
\definecolor{fillColor}{RGB}{174,153,156}

\path[fill=fillColor] ( 92.78,240.28) rectangle (289.01,248.15);
\definecolor{fillColor}{gray}{0.62}

\path[fill=fillColor] ( 92.78,248.15) rectangle (292.28,256.02);
\definecolor{drawColor}{RGB}{0,0,0}

\path[draw=drawColor,line width= 0.6pt,line join=round] (265.55, 31.63) -- (265.55,283.58);

\node[text=drawColor,anchor=base west,inner sep=0pt, outer sep=0pt, scale=  1.14] at (274.20,109.98) {7万TPS基准线};
\end{scope}
\begin{scope}
\path[clip] (  0.00,  0.00) rectangle (505.89,289.08);
\definecolor{drawColor}{gray}{0.30}

\node[text=drawColor,anchor=base east,inner sep=0pt, outer sep=0pt, scale=  0.88] at ( 73.51, 75.68) {存储模块 (存入)};

\node[text=drawColor,anchor=base east,inner sep=0pt, outer sep=0pt, scale=  0.88] at ( 73.51,154.41) {存储模块 (读取)};

\node[text=drawColor,anchor=base east,inner sep=0pt, outer sep=0pt, scale=  0.88] at ( 73.51,233.15) {执行模块};
\end{scope}
\begin{scope}
\path[clip] (  0.00,  0.00) rectangle (505.89,289.08);
\definecolor{drawColor}{gray}{0.20}

\path[draw=drawColor,line width= 0.6pt,line join=round] ( 75.71, 78.87) --
	( 78.46, 78.87);

\path[draw=drawColor,line width= 0.6pt,line join=round] ( 75.71,157.61) --
	( 78.46,157.61);

\path[draw=drawColor,line width= 0.6pt,line join=round] ( 75.71,236.34) --
	( 78.46,236.34);
\end{scope}
\begin{scope}
\path[clip] (  0.00,  0.00) rectangle (505.89,289.08);
\definecolor{drawColor}{gray}{0.20}

\path[draw=drawColor,line width= 0.6pt,line join=round] ( 92.78, 28.88) --
	( 92.78, 31.63);

\path[draw=drawColor,line width= 0.6pt,line join=round] (166.83, 28.88) --
	(166.83, 31.63);

\path[draw=drawColor,line width= 0.6pt,line join=round] (240.87, 28.88) --
	(240.87, 31.63);

\path[draw=drawColor,line width= 0.6pt,line join=round] (314.92, 28.88) --
	(314.92, 31.63);

\path[draw=drawColor,line width= 0.6pt,line join=round] (388.96, 28.88) --
	(388.96, 31.63);
\end{scope}
\begin{scope}
\path[clip] (  0.00,  0.00) rectangle (505.89,289.08);
\definecolor{drawColor}{gray}{0.30}

\node[text=drawColor,anchor=base,inner sep=0pt, outer sep=0pt, scale=  0.88] at ( 92.78, 20.29) {0};

\node[text=drawColor,anchor=base,inner sep=0pt, outer sep=0pt, scale=  0.88] at (166.83, 20.29) {3};

\node[text=drawColor,anchor=base,inner sep=0pt, outer sep=0pt, scale=  0.88] at (240.87, 20.29) {6};

\node[text=drawColor,anchor=base,inner sep=0pt, outer sep=0pt, scale=  0.88] at (314.92, 20.29) {9};

\node[text=drawColor,anchor=base,inner sep=0pt, outer sep=0pt, scale=  0.88] at (388.96, 20.29) {12};
\end{scope}
\begin{scope}
\path[clip] (  0.00,  0.00) rectangle (505.89,289.08);
\definecolor{drawColor}{RGB}{0,0,0}

\node[text=drawColor,anchor=base,inner sep=0pt, outer sep=0pt, scale=  1.10] at (235.99,  7.75) {Transaction Per Seconds TPS(万)};
\end{scope}
\begin{scope}
\path[clip] (  0.00,  0.00) rectangle (505.89,289.08);
\definecolor{fillColor}{RGB}{255,255,255}

\path[fill=fillColor] (404.51,108.10) rectangle (500.39,207.11);
\end{scope}
\begin{scope}
\path[clip] (  0.00,  0.00) rectangle (505.89,289.08);
\definecolor{drawColor}{RGB}{0,0,0}

\node[text=drawColor,anchor=base west,inner sep=0pt, outer sep=0pt, scale=  1.10] at (410.01,192.50) {基准测试};
\end{scope}
\begin{scope}
\path[clip] (  0.00,  0.00) rectangle (505.89,289.08);
\definecolor{fillColor}{gray}{0.95}

\path[fill=fillColor] (410.01,171.42) rectangle (424.47,185.87);
\end{scope}
\begin{scope}
\path[clip] (  0.00,  0.00) rectangle (505.89,289.08);
\definecolor{fillColor}{RGB}{255,108,145}

\path[fill=fillColor] (410.72,172.13) rectangle (423.76,185.16);
\end{scope}
\begin{scope}
\path[clip] (  0.00,  0.00) rectangle (505.89,289.08);
\definecolor{fillColor}{gray}{0.95}

\path[fill=fillColor] (410.01,156.96) rectangle (424.47,171.42);
\end{scope}
\begin{scope}
\path[clip] (  0.00,  0.00) rectangle (505.89,289.08);
\definecolor{fillColor}{RGB}{227,128,150}

\path[fill=fillColor] (410.72,157.67) rectangle (423.76,170.71);
\end{scope}
\begin{scope}
\path[clip] (  0.00,  0.00) rectangle (505.89,289.08);
\definecolor{fillColor}{gray}{0.95}

\path[fill=fillColor] (410.01,142.51) rectangle (424.47,156.96);
\end{scope}
\begin{scope}
\path[clip] (  0.00,  0.00) rectangle (505.89,289.08);
\definecolor{fillColor}{RGB}{199,143,153}

\path[fill=fillColor] (410.72,143.22) rectangle (423.76,156.25);
\end{scope}
\begin{scope}
\path[clip] (  0.00,  0.00) rectangle (505.89,289.08);
\definecolor{fillColor}{gray}{0.95}

\path[fill=fillColor] (410.01,128.06) rectangle (424.47,142.51);
\end{scope}
\begin{scope}
\path[clip] (  0.00,  0.00) rectangle (505.89,289.08);
\definecolor{fillColor}{RGB}{174,153,156}

\path[fill=fillColor] (410.72,128.77) rectangle (423.76,141.80);
\end{scope}
\begin{scope}
\path[clip] (  0.00,  0.00) rectangle (505.89,289.08);
\definecolor{fillColor}{gray}{0.95}

\path[fill=fillColor] (410.01,113.60) rectangle (424.47,128.06);
\end{scope}
\begin{scope}
\path[clip] (  0.00,  0.00) rectangle (505.89,289.08);
\definecolor{fillColor}{gray}{0.62}

\path[fill=fillColor] (410.72,114.31) rectangle (423.76,127.34);
\end{scope}
\begin{scope}
\path[clip] (  0.00,  0.00) rectangle (505.89,289.08);
\definecolor{drawColor}{RGB}{0,0,0}

\node[text=drawColor,anchor=base west,inner sep=0pt, outer sep=0pt, scale=  0.88] at (429.97,175.45) {第1批基准测试};
\end{scope}
\begin{scope}
\path[clip] (  0.00,  0.00) rectangle (505.89,289.08);
\definecolor{drawColor}{RGB}{0,0,0}

\node[text=drawColor,anchor=base west,inner sep=0pt, outer sep=0pt, scale=  0.88] at (429.97,161.00) {第2批基准测试};
\end{scope}
\begin{scope}
\path[clip] (  0.00,  0.00) rectangle (505.89,289.08);
\definecolor{drawColor}{RGB}{0,0,0}

\node[text=drawColor,anchor=base west,inner sep=0pt, outer sep=0pt, scale=  0.88] at (429.97,146.54) {第3批基准测试};
\end{scope}
\begin{scope}
\path[clip] (  0.00,  0.00) rectangle (505.89,289.08);
\definecolor{drawColor}{RGB}{0,0,0}

\node[text=drawColor,anchor=base west,inner sep=0pt, outer sep=0pt, scale=  0.88] at (429.97,132.09) {第4批基准测试};
\end{scope}
\begin{scope}
\path[clip] (  0.00,  0.00) rectangle (505.89,289.08);
\definecolor{drawColor}{RGB}{0,0,0}

\node[text=drawColor,anchor=base west,inner sep=0pt, outer sep=0pt, scale=  0.88] at (429.97,117.63) {第5批基准测试};
\end{scope}
\end{tikzpicture}

\end{document}

% Local Variables:
% TeX-engine: luatex
% TeX-command-extra-options: "-shell-escape"
% End: