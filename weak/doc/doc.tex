\documentclass[dvipsnames]{article}
% \documentclass[dvipsnames]{ctexrep}

\usepackage{geometry}\geometry{
  a4paper,
  total={170mm,257mm},
  left=20mm,
  top=20mm,
}

\usepackage{adjustbox}          %to narrower the caption
\newlength\mylength
\usepackage{hyperref}
\usepackage{svg}
\usepackage[skip=5pt plus1pt, indent=0pt]{parskip}
\usepackage{emoji}
% \setemojifont{NotoColorEmoji.ttf}[Path=C:/Users/congj/repo/myFonts/]
% \setemojifont{TwitterColorEmoji-SVGinOT.ttf}[Path=C:/Users/congj/repo/myFonts/]


\usepackage{booktabs}
\usepackage{tikz}
\usetikzlibrary{shapes} % ellipse node shape
\usetikzlibrary{shapes.multipart} % for line breaks in node text
\usetikzlibrary{arrows.meta}    %-o arrow head
\usetikzlibrary{arrows}
\usetikzlibrary{matrix}

\usepackage{amsmath}

\usepackage{minted}
\usepackage{tcolorbox}
\tcbuselibrary{skins}
\tcbuselibrary{minted}

\usepackage{fontspec}
\setmonofont{Cascadia}[
Path=/usr/share/fonts/truetype/Cascadia_Code/,
Scale=0.85,
Extension = .ttf,
UprightFont=*Code,              %find CascadiaCode.ttf
BoldFont=*CodePL,               %find CascadiaCodePL.ttf ...
ItalicFont=*CodeItalic,
BoldItalicFont=*CodePLItalic
]

% --------------------------------------------------
% Windows
% \setmonofont{Cascadia}[
% Path=C:/Windows/Fonts/,
% Extension = .ttf,
% UprightFont=*Mono,              %find CascadiaMono.ttf
% BoldFont=*Code,               %find CascadiaCodePL.ttf ...
% ItalicFont=*Code,
% BoldItalicFont=*Code
% ]


% Without this, there's a gap between the upper bbox and the node box

\newtcblisting{simplec}{
  listing engine=minted,
  minted language=c++,
  minted style=vs,
  minted options={fontsize=\small,autogobble,
  % framesep=1cm
  },
  tile,
  listing only,
  % bottom=0cm,
  % nobeforeafter, 
  boxsep=0mm,
  left=1mm,
  opacityback=0.5,
  colback=gray!20
}

\newcounter{myCListing}
\newtcblisting{numberedc}[3][]{
  listing engine=minted,
  minted language=c++,
  minted style=vs,
  minted options={fontsize=\small,autogobble,
  % framesep=1cm
  },
  tile,
  listing only,
  % bottom=0cm,
  % nobeforeafter, 
  boxsep=1.5mm,
  left=1mm,
  opacityback=0.5,
  colback=gray!20,
  phantom={
  \refstepcounter{myCListing}
  #3
  },
  title={代码~\themyCListing: #2},
  #1
}


\newtcblisting{simplepy}{
  listing engine=minted,
  minted language=python,
  minted style=vs,
  minted options={fontsize=\small,autogobble},
  tile,
  listing only,
  % nobeforeafter, 
  boxsep=0.5mm,
  left=1mm,
  opacityback=0.5,
  colback=gray!20,
}


\tcbset{enhanced, fontupper=\small,
  % show bounding box
}

% Color
\newcommand{\mycola}{MidnightBlue}
\newcommand{\mycolb}{Mahogany}
\newcommand{\mycolc}{OliveGreen}

\newcommand{\cola}[1]{\textcolor{\mycola}{#1}}
\newcommand{\colb}[1]{\textcolor{\mycolb}{#1}}
\newcommand{\colc}[1]{\textcolor{\mycolc}{#1}}
\newcommand{\Cola}[1]{\textcolor{\mycola}{\emph{#1}}}


\newcommand{\myTwo}[2]{\texttt{#1}为\texttt{#2}}

\tcbset{right=1mm,halign=flush left}
\newtcolorbox{ifaceBox}[1][]{colframe=\mycola, nobeforeafter, 
  fontupper=\footnotesize, left=2mm, #1}

\newtcolorbox{varBox}[1][]{nobeforeafter, 
  fontupper=\footnotesize, left=2mm, #1}

\newtcolorbox{weakBox}[1][]{colframe=gray!80, nobeforeafter,
  fontupper=\footnotesize, left=2mm, #1}

\newtcolorbox{funcBox}[1][]{colframe=\mycolb, nobeforeafter, 
  fontupper=\footnotesize, left=2mm, #1}

\newtcolorbox{noteBox}[1][]{colframe=\mycolc, nobeforeafter, 
  fontupper=\footnotesize, left=2mm, #1}

\newtcbox{\libbox}[1][\mycala]{on line,
  arc=0pt,outer arc=0pt,colback=#1!10!white,colframe=#1!50!black,
  boxsep=0pt,left=1pt,right=1pt,top=2pt,bottom=2pt,
  boxrule=0pt,bottomrule=1pt,toprule=1pt}

% copied from tcolorbox mannual
\newtcbox{\mylib}[1][\mycola]{on line,
  arc=0pt,outer arc=0pt,colback=#1!10!white,colframe=#1!50!black,
  boxsep=0pt,left=1pt,right=1pt,top=2pt,bottom=2pt,
  boxrule=0pt,bottomrule=1pt,toprule=1pt,
  left skip=0.1cm,
  right skip=0.1cm,
  fontupper=\ttfamily
}


% Redefine em
%latex.sty just do: \DeclareTextFontCommand{\emph}{\em}

\let\emph\relax % there's no \RedeclareTextFontCommand
\DeclareTextFontCommand{\emph}{\bfseries\em}

% Redefine section
\usepackage[small]{titlesec}


% curve version
% \newcommand\uptoleft[2][]{\draw[very thick,->](#1.south) to [out=270,in=180] (#2.west);}

% right-angle version
\newcommand\Southtowest[3][-o]{\draw[very thick,#1](#2.south) |- (#3.west);}
\newcommand\Southtoeast[3][-o]{\draw[very thick,#1](#2.south) |- (#3.east);}
\newcommand\Northtowest[3][-o]{\draw[very thick,#1](#2.north) |- (#3.west);}
\newcommand\Northtoeast[3][-o]{\draw[very thick,#1](#2.north) |- (#3.east);}

\newcommand\southtonorth[3][-o]{\draw[very thick,#1](#2.south) to [out=270,in=90] (#3.north);}
\newcommand\northtosouth[3][-latex]{\draw[very thick,#1](#2.north) to [out=90,in=270] (#3.south);}


\newcommand\easttowest[3][-latex]{\draw[very thick,#1](#2.east) to[out=0,in=180] (#3.west);}
\newcommand\easttonorth[3][-latex]{\draw[very thick,#1](#2.east) to[out=0,in=90] (#3.north);}
\newcommand\easttosouth[3][-latex]{\draw[very thick,#1](#2.east) to[out=0,in=270] (#3.south);}

% the left node link my ``up''

\newcommand\westTowest[3][-latex]{\draw[very thick,#1](#2.west) to[out=180,in=180] (#3.west);}
\newcommand\westTonorth[3][-latex]{\draw[very thick,#1](#2.west) to[out=180,in=90] (#3.north);}
\newcommand{\mySection}[1]{\section*{\texttt{#1}}}

\tikzstyle{every node}=[inner sep=0pt]
\tikzstyle{iface}=[rectangle,fill=gray!30,inner sep=2mm]
\tikzstyle{myMatrix}=[matrix of nodes,below right,
nodes={right},                  %apply to all nodes
row sep=1cm,column sep=2cm]

% TREE NODE
\tikzstyle{myTreeNode}=[draw=\mycola,anchor=west,inner sep=2mm,rectangle,rounded corners,fill=gray!20]
\tikzstyle{edge from parent}=[draw=\mycola,thick,-latex]
\tikzstyle{every child node}=[myTreeNode]

\usepackage{simpsons}
\usepackage{cleveref}
\crefname{figure}{图}{图}
%                    ^^^ plural
\Crefname{figure}{图}{图}
%         ^^^^^^ type = counter name

\crefname{table}{表}{表}
\Crefname{table}{表}{表}

\crefname{section}{}{}
\Crefname{section}{}{}
\creflabelformat{section}{第#2#1#3章节}


% #2 : start of hyperlink
% #3 : end of hyperlink
% #1 : The counter

% U need to define both crefname and Crefname , else it will cause fatal error.
\crefname{myCListing}{代码}{代码}
\Crefname{myCListing}{代码}{代码}

\title{Weak Chain User Manual}
\date{\today}
\author{cccccje}
% \usepackage{fontspec}

\usepackage{fancyhdr}
\fancyhf{}                      % clear all header and footer fields
\lhead{\textnormal{\rightmark}}
% \rhead{--\ \thepage\ --}
\rfoot{--\ \thepage\ --}
\pagestyle{fancy}

\fancyfoot[C]{
  \begin{tikzpicture}[remember picture, overlay]
    % draw a corner at the top right
    \draw[fill=gray,draw=none,opacity=0.2] (current page.north east) -- +(-2,0) -- +(0,-6) -- cycle;
    \draw[fill=gray,draw=none,opacity=0.7] (current page.north east) -- +(-3,0) -- +(0,-4) -- cycle;
    \draw[fill=\mycola,draw=none,opacity=0.7] (current page.north east) -- +(-7,0) -- +(0,-3) -- cycle;


    % draw at the bottom left
    \draw[fill=gray,draw=none,opacity=0.6] (current page.south west) --
    +(7,0) -- +(0,2) -- cycle;
    \draw[fill=\mycola,draw=none,opacity=0.6] (current page.south west) --
    +(2,0) -- +(0,6) -- cycle;
    \draw[fill=gray,draw=none,opacity=0.3] (current page.south west) --
    +(3,0) -- +(0,4) -- cycle;
    
  \end{tikzpicture}
}

\newcommand{\Wch}{\strong{\cola{Weak Chain}}}

\begin{document}
\maketitle
\tableofcontents{}
\newpage{}
\section{Welcome}
This is the user manual for \Wch{}. \Wch{} is a private
blockchain. The name come from the idea that
  \cola{
  ``you only need a weak understanding of blockchain
  in order to use it''
}. This manual includes some concepts and some tutorials.
It's hoped that following this manual, you will be able to get the most from its
features.

In the hope of improving readability, this manual is written as dialogues
between our two characters, Coco the Parrot \emoji{parrot} and Tim the Turtle
\emoji{turtle}. Coco is a student. He is curious but not very patient. He just
wants to get things done, but he's not a big fan of becoming a blockchain
expert. Tim is the teacher. He is more knowledgeable, and willing to share
everything he knows, but meanwhile keeping things as simple as possible. Tim
often get pissed off by Coco's impatience, but he knows that's just what he
should deal with in order to be a good teacher.

If you are ready, join the journey with Coco and Tim. Let's get started.

\section{Concepts}

\colz{
  \emoji{parrot} : I don't know much about blockchain, but I did hear of
  cryptocurrency like Bitcoin. Is \Wch{} something like that?
}

\emoji{turtle} : Not quite. Blockchain is like a family. It has many members.
Cryptocurrency, like Bitcoin, is one of them. The members share some common
concepts and ideas including:

\begin{enumerate}
\item Consensus mechanisms
\item Ledger as database
\item etc...
\end{enumerate}

\Wch{} belongs one of the members called \cola{private chain}. It's called
``private'', because in contrast to public chain like Cryptocurrency, the nodes
are often run in private networks, and the data is not open to the public. It
can do things such as making sure that the data is not tampered with unless a
majority of the nodes are controlled by the same person. As a result, it can be used to
store things like contracts.

Private chain has its own core concepts. Some are shared with other members, and
some are not. These differences however, make some members quite different from
each other, even though they are all called ``blockchain''. It's like penguin
and hummingbird are both ``birds'', but they are quite different.

In this section we will go through some of the important concepts of
private chain. 

\emoji{parrot} : Okay... But I don't wanna know too much details. I just wanna
get the chain running and use it.

\emoji{turtle} : If you just want to use \Wch{}, skip to \cref{sec:tut} to see
the tutorial. You can always come back here later if you find some scary terms.

\emoji{parrot} : Okay.

\begin{weakBox}[title=Some terms]
  \emoji{turtle} : From now on, when we say ``blockchain'', we mean all the
  whole blockchain family, and when we say ``private blockchain'' or simply
  ``private chain'', we mean the member that \Wch{} belongs to.
\end{weakBox}

\subsection{Storage and Execution of Transactions}
Blockchain is a shared database.


\section{Tutorial}
\label{sec:tut}


\end{document}

% Local Variables:
% TeX-engine: luatex
% TeX-command-extra-options: "-shell-escape"
% TeX-master: "m.tex"
% TeX-parse-self: t
% TeX-auto-save: t
% End: