\usepackage{tikz}
\usetikzlibrary{shapes} % ellipse node shape
\usetikzlibrary{shapes.multipart} % for line breaks in node text
\usetikzlibrary{arrows.meta}    %-o arrow head
\usetikzlibrary{arrows}
\usetikzlibrary{matrix}

\usepackage{amsmath}

\usepackage{minted}
\usepackage{tcolorbox}
\tcbuselibrary{skins}
\tcbuselibrary{minted}

\usepackage{fontspec}
\setmonofont{Cascadia}[
Path=/usr/share/fonts/truetype/Cascadia_Code/,
Scale=0.85,
Extension = .ttf,
UprightFont=*Code,              %find CascadiaCode.ttf
BoldFont=*CodePL,               %find CascadiaCodePL.ttf ...
ItalicFont=*CodeItalic,
BoldItalicFont=*CodePLItalic
]

% --------------------------------------------------
% Windows
% \setmonofont{Cascadia}[
% Path=C:/Windows/Fonts/,
% Extension = .ttf,
% UprightFont=*Mono,              %find CascadiaMono.ttf
% BoldFont=*Code,               %find CascadiaCodePL.ttf ...
% ItalicFont=*Code,
% BoldItalicFont=*Code
% ]


% Without this, there's a gap between the upper bbox and the node box

\newtcblisting{simplec}{
  listing engine=minted,
  minted language=c++,
  minted style=vs,
  minted options={fontsize=\small,autogobble,
  % framesep=1cm
  },
  tile,
  listing only,
  % bottom=0cm,
  % nobeforeafter, 
  boxsep=0mm,
  left=1mm,
  opacityback=0.5,
  colback=gray!20
}

\newcounter{myCListing}
\newtcblisting{numberedc}[3][]{
  listing engine=minted,
  minted language=c++,
  minted style=vs,
  minted options={fontsize=\small,autogobble,
  % framesep=1cm
  },
  tile,
  listing only,
  % bottom=0cm,
  % nobeforeafter, 
  boxsep=1.5mm,
  left=1mm,
  opacityback=0.5,
  colback=gray!20,
  phantom={
  \refstepcounter{myCListing}
  #3
  },
  title={代码~\themyCListing: #2},
  #1
}


\newtcblisting{simplepy}{
  listing engine=minted,
  minted language=python,
  minted style=vs,
  minted options={fontsize=\small,autogobble},
  tile,
  listing only,
  % nobeforeafter, 
  boxsep=0.5mm,
  left=1mm,
  opacityback=0.5,
  colback=gray!20,
}


\tcbset{enhanced, fontupper=\small,
  % show bounding box
}

% Color
\newcommand{\mycola}{MidnightBlue}
\newcommand{\mycolb}{Mahogany}
\newcommand{\mycolc}{OliveGreen}

\newcommand{\cola}[1]{\textcolor{\mycola}{#1}}
\newcommand{\colb}[1]{\textcolor{\mycolb}{#1}}
\newcommand{\colc}[1]{\textcolor{\mycolc}{#1}}
\newcommand{\Cola}[1]{\textcolor{\mycola}{\emph{#1}}}


\newcommand{\myTwo}[2]{\texttt{#1}为\texttt{#2}}

\tcbset{right=1mm,halign=flush left}
\newtcolorbox{ifaceBox}[1][]{colframe=\mycola, nobeforeafter, 
  fontupper=\footnotesize, left=2mm, #1}

\newtcolorbox{varBox}[1][]{nobeforeafter, 
  fontupper=\footnotesize, left=2mm, #1}

\newtcolorbox{weakBox}[1][]{colframe=gray!80, nobeforeafter,
  fontupper=\footnotesize, left=2mm, #1}

\newtcolorbox{funcBox}[1][]{colframe=\mycolb, nobeforeafter, 
  fontupper=\footnotesize, left=2mm, #1}

\newtcolorbox{noteBox}[1][]{colframe=\mycolc, nobeforeafter, 
  fontupper=\footnotesize, left=2mm, #1}

\newtcbox{\libbox}[1][\mycala]{on line,
  arc=0pt,outer arc=0pt,colback=#1!10!white,colframe=#1!50!black,
  boxsep=0pt,left=1pt,right=1pt,top=2pt,bottom=2pt,
  boxrule=0pt,bottomrule=1pt,toprule=1pt}

% copied from tcolorbox mannual
\newtcbox{\mylib}[1][\mycola]{on line,
  arc=0pt,outer arc=0pt,colback=#1!10!white,colframe=#1!50!black,
  boxsep=0pt,left=1pt,right=1pt,top=2pt,bottom=2pt,
  boxrule=0pt,bottomrule=1pt,toprule=1pt,
  left skip=0.1cm,
  right skip=0.1cm,
  fontupper=\ttfamily
}


% Redefine em
%latex.sty just do: \DeclareTextFontCommand{\emph}{\em}

\let\emph\relax % there's no \RedeclareTextFontCommand
\DeclareTextFontCommand{\emph}{\bfseries\em}

% Redefine section
\usepackage[small]{titlesec}


% curve version
% \newcommand\uptoleft[2][]{\draw[very thick,->](#1.south) to [out=270,in=180] (#2.west);}

% right-angle version
\newcommand\Southtowest[3][-o]{\draw[very thick,#1](#2.south) |- (#3.west);}
\newcommand\Southtoeast[3][-o]{\draw[very thick,#1](#2.south) |- (#3.east);}
\newcommand\Northtowest[3][-o]{\draw[very thick,#1](#2.north) |- (#3.west);}
\newcommand\Northtoeast[3][-o]{\draw[very thick,#1](#2.north) |- (#3.east);}

\newcommand\southtonorth[3][-o]{\draw[very thick,#1](#2.south) to [out=270,in=90] (#3.north);}
\newcommand\northtosouth[3][-latex]{\draw[very thick,#1](#2.north) to [out=90,in=270] (#3.south);}


\newcommand\easttowest[3][-latex]{\draw[very thick,#1](#2.east) to[out=0,in=180] (#3.west);}
\newcommand\easttonorth[3][-latex]{\draw[very thick,#1](#2.east) to[out=0,in=90] (#3.north);}
\newcommand\easttosouth[3][-latex]{\draw[very thick,#1](#2.east) to[out=0,in=270] (#3.south);}

% the left node link my ``up''

\newcommand\westTowest[3][-latex]{\draw[very thick,#1](#2.west) to[out=180,in=180] (#3.west);}
\newcommand\westTonorth[3][-latex]{\draw[very thick,#1](#2.west) to[out=180,in=90] (#3.north);}
\newcommand{\mySection}[1]{\section*{\texttt{#1}}}

\tikzstyle{every node}=[inner sep=0pt]
\tikzstyle{iface}=[rectangle,fill=gray!30,inner sep=2mm]
\tikzstyle{myMatrix}=[matrix of nodes,below right,
nodes={right},                  %apply to all nodes
row sep=1cm,column sep=2cm]

% TREE NODE
\tikzstyle{myTreeNode}=[draw=\mycola,anchor=west,inner sep=2mm,rectangle,rounded corners,fill=gray!20]
\tikzstyle{edge from parent}=[draw=\mycola,thick,-latex]
\tikzstyle{every child node}=[myTreeNode]

\usepackage{simpsons}
\usepackage{cleveref}
\crefname{figure}{图}{图}
%                    ^^^ plural
\Crefname{figure}{图}{图}
%         ^^^^^^ type = counter name

\crefname{table}{表}{表}
\Crefname{table}{表}{表}

\crefname{section}{}{}
\Crefname{section}{}{}
\creflabelformat{section}{第#2#1#3章节}


% #2 : start of hyperlink
% #3 : end of hyperlink
% #1 : The counter

% U need to define both crefname and Crefname , else it will cause fatal error.
\crefname{myCListing}{代码}{代码}
\Crefname{myCListing}{代码}{代码}