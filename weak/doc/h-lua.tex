
% Color
\newcommand{\mycola}{Purple}
\newcommand{\mycolb}{Mahogany}
\newcommand{\mycolc}{OliveGreen}

% \newcommand{\cola}[1][]{\textcolor{\mycola}{#1}}
% 🦜 : ⚠️ : For this we need to use \cola[hi] instead of \cola{hi}...

\newcommand{\cola}[1]{\textcolor{\mycola}{#1}}
\newcommand{\colb}[1]{\textcolor{\mycolb}{#1}}
\newcommand{\colc}[1]{\textcolor{\mycolc}{#1}}
\newcommand{\colz}[1]{\textcolor{gray}{#1}}
\newcommand{\Cola}[1]{\textcolor{\mycola}{\textbf{#1}}}

\usepackage{tikz}
\usetikzlibrary{shapes} % ellipse node shape
\usetikzlibrary{shapes.multipart} % for line breaks in node text
\usetikzlibrary{arrows.meta}    %-o arrow head
\usetikzlibrary{arrows}
\usetikzlibrary{matrix}

\usepackage{amsmath}

\usepackage{minted}
\usepackage{tcolorbox}
\tcbuselibrary{skins}
\tcbuselibrary{minted}
\tcbuselibrary{breakable}

\usepackage{fontspec}
\setmonofont{Cascadia}[
Path=/usr/share/fonts/truetype/Cascadia_Code/,
Scale=0.85,
Extension = .ttf,
UprightFont=*Code,              %find CascadiaCode.ttf
BoldFont=*CodePL,               %find CascadiaCodePL.ttf ...
ItalicFont=*CodeItalic,
BoldItalicFont=*CodePLItalic
]
\usepackage{amsmath}

\usepackage{minted}
\usepackage{tcolorbox}
\tcbuselibrary{skins}
\tcbuselibrary{minted}

\usepackage{fontspec}
\setmonofont{Cascadia}[
Path=/usr/share/fonts/truetype/Cascadia_Code/,
Scale=0.85,
Extension = .ttf,
UprightFont=*Code,              %find CascadiaCode.ttf
BoldFont=*CodePL,               %find CascadiaCodePL.ttf ...
ItalicFont=*CodeItalic,
BoldItalicFont=*CodePLItalic
]



% Without this, there's a gap between the upper bbox and the node box

\newcommand{\makeSimpleListing}[2]{
  \newtcblisting{#1}{
    listing engine=minted,
    minted language=#2,
    minted style=vs,
    minted options={fontsize=\small,autogobble,
      breaklines=true,
      breakanywhere=true
      % framesep=1cm
    },
    tile,
    listing only,
    % bottom=0cm,
    % nobeforeafter,
    boxsep=0mm,
    left=1mm,
    opacityback=0.5,
    colback=gray!20,
    breakable
  }
}

% pygmentize -L lexers
\makeSimpleListing{simplec}{c++}
\makeSimpleListing{simplepy}{python}
\makeSimpleListing{simpleyml}{yaml}
\makeSimpleListing{simplesh}{bash}
\makeSimpleListing{simplecf}{cfg}
\makeSimpleListing{simplexml}{xml}
\makeSimpleListing{simplejv}{java}
\makeSimpleListing{simplejs}{js}
\makeSimpleListing{simplesol}{solidity}


\tcbset{enhanced, fontupper=\small,
  % show bounding box
}



\newcommand{\myTwo}[2]{\texttt{#1}为\texttt{#2}}

\tcbset{right=1mm,halign=flush left}
\newtcolorbox{ifaceBox}[1][]{colframe=\mycola, nobeforeafter, 
  fontupper=\footnotesize, left=2mm, #1}

\newtcolorbox{varBox}[1][]{nobeforeafter, 
  fontupper=\footnotesize, left=0mm, #1}

\newtcolorbox{weakBox}[1][]{colframe=gray!80, nobeforeafter,
  fontupper=\footnotesize, left=0mm, #1}

\newtcolorbox{funcBox}[1][]{colframe=\mycolb, nobeforeafter, 
  fontupper=\footnotesize, left=0mm, #1}

\newtcolorbox{noteBox}[1][]{colframe=\mycolc, nobeforeafter, 
  fontupper=\footnotesize, left=0mm, #1}

\newtcbox{\libbox}[1][\mycala]{on line,
  arc=0pt,outer arc=0pt,colback=#1!10!white,colframe=#1!50!black,
  boxsep=0pt,left=1pt,right=1pt,top=2pt,bottom=2pt,
  boxrule=0pt,bottomrule=1pt,toprule=1pt}

% copied from tcolorbox mannual
\newtcbox{\mylib}[1][\mycola]{on line,
  arc=0pt,outer arc=0pt,colback=#1!10!white,colframe=#1!50!black,
  boxsep=0pt,left=1pt,right=1pt,top=2pt,bottom=2pt,
  boxrule=0pt,bottomrule=1pt,toprule=1pt,
  left skip=0.1cm,
  right skip=0.1cm,
  fontupper=\ttfamily
}


% Redefine em
%latex.sty just do: \DeclareTextFontCommand{\emph}{\em}

\let\emph\relax % there's no \RedeclareTextFontCommand
\DeclareTextFontCommand{\emph}{\bfseries\em}

% Redefine section
\usepackage[small]{titlesec}


% curve version
% \newcommand\uptoleft[2][]{\draw[very thick,->](#1.south) to [out=270,in=180] (#2.west);}

% right-angle version
\newcommand\uptoleft[3][-o]{\draw[very thick,#1](#2.south) |- (#3.west);}
\newcommand\uptodown[3][-o]{\draw[very thick,#1](#2.south) to [out=270,in=90] (#3.north);}
\newcommand\downtoup[3][-latex]{\draw[very thick,#1](#2.north) to [out=90,in=270] (#3.south);}

\newcommand\lefttoright[3][-latex]{\draw[very thick,#1](#2.east) to[out=0,in=180] (#3.west);}
\newcommand\lefttodown[3][-latex]{\draw[very thick,#1](#2.east) to[out=0,in=90] (#3.north);}
\newcommand{\mySection}[1]{\section*{\texttt{#1}}}

\tikzstyle{every node}=[inner sep=0pt]
\tikzstyle{iface}=[rectangle,fill=gray!30,inner sep=2mm]
\tikzstyle{myMatrix}=[matrix of nodes,below right,
nodes={right},                  %apply to all nodes
row sep=1cm,column sep=2cm]

% TREE NODE
\tikzstyle{myTreeNode}=[draw=\mycola,anchor=west,inner sep=2mm,rectangle,rounded corners,fill=gray!20]
\tikzstyle{edge from parent}=[draw=\mycola,thick,-latex]
\tikzstyle{every child node}=[myTreeNode]

\usepackage{simpsons}


\newenvironment{aBox}
  {
\begin{tcolorbox}[bicolor,sidebyside,
 lefthand width=0.8cm, tile,
  sharp corners,boxrule=.4pt,colback=gray!20,colbacklower=\mycolb!20,
  fontupper=\footnotesize,
  fontlower=\footnotesize,
  left=0cm,
  ]
\Bart{}
\tcblower{}
% \tcbfontsize{0.5}
}{\end{tcolorbox}}

\newenvironment{bBox}
  {
\begin{tcolorbox}[bicolor,sidebyside,
 lefthand width=0.8cm, tile,
  sharp corners,boxrule=.4pt,colback=gray!20,colbacklower=\mycola!20,
  fontupper=\footnotesize,
  fontlower=\footnotesize,
  left=0cm,
  ]
\Lisa{}
\tcblower{}
% \tcbfontsize{0.5}
}{\end{tcolorbox}}

\newenvironment{cBox}
  {
\begin{tcolorbox}[bicolor,sidebyside,
 lefthand width=0.8cm, tile,
  sharp corners,boxrule=.4pt,colback=gray!20,colbacklower=\mycola!20,
  fontupper=\footnotesize,
  fontlower=\footnotesize,
  left=0cm,
  ]
\SNPP{}
\tcblower{}
% \tcbfontsize{0.5}
}{\end{tcolorbox}}
% --------------------------------------------------
% Windows
% \setmonofont{Cascadia}[
% Path=C:/Windows/Fonts/,
% Extension = .ttf,
% UprightFont=*Mono,              %find CascadiaMono.ttf
% BoldFont=*Code,               %find CascadiaCodePL.ttf ...
% ItalicFont=*Code,
% BoldItalicFont=*Code
% ]


% \tcbset{enhanced, fontupper=\small,
%   % show bounding box
% }

% % Color
% \newcommand{\mycola}{MidnightBlue}
% \newcommand{\mycolb}{Mahogany}
% \newcommand{\mycolc}{OliveGreen}

% \newcommand{\cola}[1]{\textcolor{\mycola}{#1}}
% \newcommand{\colb}[1]{\textcolor{\mycolb}{#1}}
% \newcommand{\colc}[1]{\textcolor{\mycolc}{#1}}
% \newcommand{\Cola}[1]{\textcolor{\mycola}{\emph{#1}}}

% % 🦜 : \textcolor doesn't allow multiple paragraphs in it, so we used {\color{...}}
% \newcommand{\colZ}[1]{
% {\color{black}#1}
% } %go back
% \newcommand{\colz}[1]{
% {\color{gray}#1}
% }

% \newcommand{\myTwo}[2]{\texttt{#1}为\texttt{#2}}

% \tcbset{right=1mm,halign=flush left}
% \newtcolorbox{ifaceBox}[1][]{colframe=\mycola, nobeforeafter, 
%   fontupper=\footnotesize, left=2mm, #1}

% \newtcolorbox{varBox}[1][]{nobeforeafter, 
%   fontupper=\footnotesize, left=2mm, #1}

% \newtcolorbox{weakBox}[1][]{colframe=gray!80, nobeforeafter,
%   fontupper=\footnotesize, left=2mm, #1}

% \newtcolorbox{funcBox}[1][]{colframe=\mycolb, nobeforeafter, 
%   fontupper=\footnotesize, left=2mm, #1}

% \newtcolorbox{noteBox}[1][]{colframe=\mycolc, nobeforeafter, 
%   fontupper=\footnotesize, left=2mm, #1}

% \newtcbox{\libbox}[1][\mycala]{on line,
%   arc=0pt,outer arc=0pt,colback=#1!10!white,colframe=#1!50!black,
%   boxsep=0pt,left=1pt,right=1pt,top=2pt,bottom=2pt,
%   boxrule=0pt,bottomrule=1pt,toprule=1pt}

% % copied from tcolorbox mannual
% \newtcbox{\mylib}[1][\mycola]{on line,
%   arc=0pt,outer arc=0pt,colback=#1!10!white,colframe=#1!50!black,
%   boxsep=0pt,left=1pt,right=1pt,top=2pt,bottom=2pt,
%   boxrule=0pt,bottomrule=1pt,toprule=1pt,
%   left skip=0.1cm,
%   right skip=0.1cm,
%   fontupper=\ttfamily
% }


% % Redefine em
% %latex.sty just do: \DeclareTextFontCommand{\emph}{\em}

% \let\emph\relax % there's no \RedeclareTextFontCommand
% \DeclareTextFontCommand{\emph}{\bfseries\em}

% % Redefine section
% \usepackage[small]{titlesec}


% % curve version
% % \newcommand\uptoleft[2][]{\draw[very thick,->](#1.south) to [out=270,in=180] (#2.west);}

% % right-angle version
% \newcommand\Southtowest[3][-o]{\draw[very thick,#1](#2.south) |- (#3.west);}
% \newcommand\Southtoeast[3][-o]{\draw[very thick,#1](#2.south) |- (#3.east);}
% \newcommand\Northtowest[3][-o]{\draw[very thick,#1](#2.north) |- (#3.west);}
% \newcommand\Northtoeast[3][-o]{\draw[very thick,#1](#2.north) |- (#3.east);}

% \newcommand\southtonorth[3][-o]{\draw[very thick,#1](#2.south) to [out=270,in=90] (#3.north);}
% \newcommand\northtosouth[3][-latex]{\draw[very thick,#1](#2.north) to [out=90,in=270] (#3.south);}


% \newcommand\easttowest[3][-latex]{\draw[very thick,#1](#2.east) to[out=0,in=180] (#3.west);}
% \newcommand\easttonorth[3][-latex]{\draw[very thick,#1](#2.east) to[out=0,in=90] (#3.north);}
% \newcommand\easttosouth[3][-latex]{\draw[very thick,#1](#2.east) to[out=0,in=270] (#3.south);}

% % the left node link my ``up''

% \newcommand\westTowest[3][-latex]{\draw[very thick,#1](#2.west) to[out=180,in=180] (#3.west);}
% \newcommand\westTonorth[3][-latex]{\draw[very thick,#1](#2.west) to[out=180,in=90] (#3.north);}
% \newcommand{\mySection}[1]{\section*{\texttt{#1}}}

% \tikzstyle{every node}=[inner sep=0pt]
% \tikzstyle{iface}=[rectangle,fill=gray!30,inner sep=2mm]
% \tikzstyle{myMatrix}=[matrix of nodes,below right,
% nodes={right},                  %apply to all nodes
% row sep=1cm,column sep=2cm]

% % TREE NODE
% \tikzstyle{myTreeNode}=[draw=\mycola,anchor=west,inner sep=2mm,rectangle,rounded corners,fill=gray!20]
% \tikzstyle{edge from parent}=[draw=\mycola,thick,-latex]
% \tikzstyle{every child node}=[myTreeNode]

% \usepackage{simpsons}

% \usepackage{hyperref}
% \hypersetup{
%   colorlinks=true,
%   linkcolor=\mycola,
%   filecolor=\mycola,      
%   urlcolor=\mycola,
%   pdftitle={Weak-chain memp},
%   % pdfpagemode=FullScreen,
% }
% \urlstyle{same}
% % 🦜 : cleveref must be loaded after hyperref
% \usepackage{cleveref}
% \crefname{figure}{图}{图}
% %                    ^^^ plural
% \Crefname{figure}{图}{图}
% %         ^^^^^^ type = counter name

% \crefname{table}{表}{表}
% \Crefname{table}{表}{表}

% \crefname{section}{}{}
% \Crefname{section}{}{}
% \creflabelformat{section}{第#2#1#3章节}


% % #2 : start of hyperlink
% % #3 : end of hyperlink
% % #1 : The counter

% % U need to define both crefname and Crefname , else it will cause fatal error.
% \crefname{myCListing}{代码}{代码}
% \Crefname{myCListing}{代码}{代码}

