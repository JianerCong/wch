\section{摘要}

这篇文档介绍存储和执行模块。我们将从抽象到形象,底层到上层,依次介绍接口和类
\begin{figure}[h]
  \begin{center}
    \begin{tikzpicture}
      % \draw[style=help lines] (0cm,0cm) grid +(15cm,-10cm);
      \node (WorldStorage) [text width=6cm] at (2cm,-2cm)
      {\begin{ifaceBox}[]
          % \big
          \large
          \textbf{区块链数据World Storage}
        \end{ifaceBox}};
      
      \matrix (M-WorldStorage) [myMatrix,
      shift={(2,-2)},
      text width=12cm]
      at (WorldStorage.center){
        \raisebox{-.3\height}{\includesvg[width=0.5cm]{img/database.svg}}
        % ------- the height of the to-be raised box
        \cola{账本}: 重点包括一个个已经执行了的区块 (“整个链”),以及其他信息。因
        此有些时候我们也说 “\cola{链} + \colb{状态}” 而不是 “\cola{账本} + \colb{状态}”
        \\
        \raisebox{-.3\height}{\includesvg[width=0.5cm]{img/database.svg}} \colb{状
         态}: 当前\colb{世界状态 World State},每次区块数增加时被改变。
        \\
      }; \foreach \i in {1,2}{ \Southtowest{WorldStorage}{M-WorldStorage-\i-1} }
    \end{tikzpicture}
  \end{center}
  \caption{广义的区块链世界数据}
  \label{fig:worldStorage}
\end{figure}

而这个\colb{世界状态}其实就是一个KV表,见\cref{fig:worldState}其中
\begin{itemize}
\item Key 为 \colc{地址 Address}, 20-bytes 字符串
\item Value 为 \cola{账户 Account},象征着每一个每一个部署的合约或用户。
\end{itemize}
\begin{figure}[h]
  \begin{center}
    \begin{ifaceBox}[width=\linewidth/2,title=\texttt{WorldState}]
      \begin{center}
        \begin{tikzpicture}
          \matrix (M1) [matrix of nodes,column sep=2cm, row sep=0.3cm, nodes={ellipse,
            text height=1.3em,
            text depth=1.1em,
            text centered
          }]{
            |[draw,text=\mycolc]| \texttt{address-1} & |[draw,text=\mycola]| \texttt{account-1} \\
            |[draw,text=\mycolc]| \texttt{address-2} & |[draw,text=\mycola]| \texttt{account-2} \\
            |[draw,text=\mycolc]| \texttt{address-3} & |[draw,text=\mycola]| \texttt{account-3} \\
            \texttt{...} & \texttt{...} \\
          };
          \foreach \i in {1,2,3}{
            \easttowest{[xshift=.5cm]M1-\i-1}{[xshift=-.5cm]M1-\i-2}
          }
        \end{tikzpicture}
      \end{center}
    \end{ifaceBox}
  \end{center}
  \caption{世界状态World State内部是address到account的KV表}
  \label{fig:worldState}
\end{figure}

\section{底层数据结构}

\subsection{账户 Account}

而这里我们就要介绍非常重要的\cola{账户Account}数据结构。
\begin{tcolorbox}
  \emoji{parrot} : 要按重要性把区块链世界里的数据结构来排名的
  话,我觉得\cola{账户Account}起码能排前三。
\end{tcolorbox}
Account,除了一些其他辅助类信息(如\texttt{nonce},\texttt{hash},...),重点包含以
下两个数据:
\begin{itemize}
\item \texttt{storage} 存储: 一个\texttt{bytes32} 到 \texttt{bytes32}的 KV 表。合
  约会往这里读写数据。
\item \texttt{code} 代码: 该账户上的字节码,应该是一个可以被塞进虚拟机里执行的。
\end{itemize}
账户分为两种:
\begin{enumerate}
\item \cola{合约账户} : 代表着一个被部署了的合约,其中\texttt{code}为调用该合约的
  字节码。该字解码要配合一个ABI字符串来使用(以后说)。
\item  \colb{非合约账户}: 一般代表一个用户,没有必要有\texttt{code}。
\end{enumerate}

\begin{tcolorbox}
  \emoji{parrot} : 其实传统的Account里都还会有一个\texttt{value}代表着账户余额,
  但我们不需要,也就不提了。
\end{tcolorbox}

那么我们来看代码, 见\cref{code:acn}\footnote{上面
  的\texttt{IJsonizable}和\texttt{ISerializable}是两个描述行为(behavior)的虚接
  口。对于了解我们的数据结构而言并不重要,暂时可以先不理解我们以后再说(见\cref{sec:iface})。不过如果
  好奇的话,大概意思就是说:“这个类可以变成JSON”,“这个类可以被序列化/反序列化”。}

\begin{numberedc}{账户}{\label{code:acn}}
class Acn: virtual public IJsonizable
         , virtual public ISerializable{
public:
  Acn() = default;
  Acn(const uint64_t n,bytes c): nonce(n), code(c){
    // get hash from code
    codehash = ethash::keccak256(reinterpret_cast<const uint8_t*>(c.data()),
                                 c.size());
  };
  uint64_t nonce = 0; /// The account nonce.
  bytes code;
  hash256 codehash;
  unordered_map<bytes32,bytes32> storage;
  //... 略
}
\end{numberedc}

\begin{tcolorbox}
  \emoji{turtle} : 以上所有的类都来自于\texttt{evmc}和\texttt{ethash}:
  \begin{simplec}
    using evmc::bytes;
    using evmc::bytes_view;
    using evmc::address;
    using evmc::bytes32;
    using ethash::hash256;
  \end{simplec}
  使用模块现有的类是为了最小化代码量(也最小化bug可能性)。
\end{tcolorbox}
可以看到我们现在最简化的Account里只有4个成员:
\texttt{nonce,code,codehash,storage}。只需要这些成员就可以和EVMC进行交互了。

如果把每一个Account想象成一个小U盘的话,里面的\cola{\texttt{code}}就是里面
的\cola{程序},而里面的\colc{\texttt{storage}}就是里面的\colc{文件}。而\colb{虚拟
  机},则像是一台(没有存储的)\colb{电脑}。而调用某个合约其实就像是在虚拟机接上
这个小U盘,然后跑一跑里面的程序。具体怎么跑则由用户来决定(这个被写在ABI里)见\cref{fig:vm-abi-acn-relationship}。

\begin{figure}[h]
  \begin{center}
    \begin{tikzpicture}
      % \draw[style=help lines] (-5cm,5cm) grid +(15cm,-10cm);
      \newdimen\myX
      \myX=4cm
      
      \path node[name=VM] {
        \raisebox{-.3\height}{\includesvg[width=2cm]{img/drive.svg}} 虚拟机VM
      } -- +(-\myX,-\myX)
      node[name=ABI]{
        \raisebox{-.3\height}{\includesvg[width=2cm]{img/keyboard.svg}} ABI
      } -- +(\myX,-\myX) node[name=Acn]{
        \raisebox{-.3\height}{\includesvg[width=1cm]{img/usb.svg}} 合约账户Account
      };

      \draw[-latex,thick] (ABI.east) ..controls + (4cm,0) ..
      node[pos=0.1,sloped,above] {指定}
      node[pos=0.2,sloped,above] {合约}
      node[pos=0.35,sloped,above] {调用}
      (VM);

      \draw[-latex,thick] ([xshift=0.5cm]VM.east) --
      node[pos=0.5,sloped,above] {改变内部存储}
      (Acn.north);
    \end{tikzpicture}
  \end{center}
  \par% or empty line, needed to get caption below the image, not to the rigth
  \adjustbox{minipage=12cm,center}{
    \caption{虚拟机,用户ABI以及合约账户之间关系的一个比喻:虚拟机不储存状态就像一
      个CPU一样,状态和内存都存在账户里,而用户提供的ABI会调用账户里的\cola{程序}来
      改变帐户里的\colc{文件}。}
  }
  \label{fig:vm-abi-acn-relationship}
\end{figure}

\subsection{交易 Transaction}
\label{sec:tx}
接下来我们介绍同样重要的 \cola{交易} 数据结构。外部世界通过交易来对区块链世界的状态进行改变,如\cref{fig:tx-world}。

\begin{figure}[h]
  \centering
  \begin{tikzpicture}
    % \draw[style=help lines] (-2cm,5cm) grid +(15cm,-10cm);
    \node[name=globe]{
      \raisebox{-.3\height}{\includesvg[width=2cm]{img/globe2.svg}}
    };
    \node at ([yshift=2cm]globe){外部世界};

    \matrix (M1) [matrix of nodes,column sep=2cm,row sep=2cm] at (10cm,0cm){
      \raisebox{-.3\height}{\includesvg[width=1cm]{img/hdd-stack-fill.svg}}
      &\raisebox{-.3\height}{\includesvg[width=1cm]{img/hdd-stack-fill.svg}}
      \\
        \raisebox{-.3\height}{\includesvg[width=1cm]{img/hdd-stack-fill.svg}}
      &\raisebox{-.3\height}{\includesvg[width=1cm]{img/hdd-stack-fill.svg}}
      \\
    };

    \node at ([yshift=3cm]M1){区块链集群};

    \tikzstyle{myArrow} = [thick,latex-latex,shorten >=10pt, shorten <=10pt,draw=\mycola]
    %                                        --- shortens the start of line
    \draw[myArrow] (M1-1-1) -- (M1-1-2);
    \draw[myArrow] (M1-1-1) -- (M1-2-1);
    \draw[myArrow] (M1-2-1) -- (M1-2-2);
    \draw[myArrow] (M1-1-2) -- (M1-2-2);

    % The Tx ->
    \draw[-latex,thick,shorten >=10pt, shorten <=10pt,color=gray] (globe) -- (M1) node[midway,above,yshift=2pt]{
      发起\cola{交易}改变区块链\colb{状态}。
    }
    node[midway,above,yshift=0.7cm]{
      \raisebox{-.3\height}{\includesvg[width=0.9cm]{img/envelope.svg}}
    }
    ;
    
  \end{tikzpicture}
  \caption{外部世界通过交易对区块链世界状态进行改变}
  \label{fig:tx-world}
\end{figure}

每条交易包括:
\begin{itemize}
\item 发起方: \texttt{from}
\item 接收方: \texttt{to}
\item 交易数据: \texttt{data}
\item 交易序列号(UUID) :\texttt{nonce}
\item 交易进入区块链集群的时间戳: \texttt{timestamp}
\item 交易哈希: \texttt{hash}
\end{itemize}

具体实现见\cref{code:tx}。

\begin{numberedc}{交易}{\label{code:tx}}
 class Tx: virtual public IJsonizable
        , virtual public ISerializable{
public:
  Tx() = default; // Upon you make a constructor yourself, the build-in
  // constructor is gone,use this to bring it back

  Tx(const address f,const address t,const bytes d,const uint64_t n):
    from(f),to(t),data(d),nonce(n){
    timestamp = std::time(nullptr);
    // convert nonce to array
    // ...计算交易hash
  };

  address from;
  /// if to is empty, create contract, else call
  address  to;
  /// if CREATE: contract bin code, else abi input
  bytes data;
  /// unique_id
  uint64_t nonce;
  /// ms since Epoch
  std::time_t timestamp;
  // usually = std::time(nullptr);
  /// Tx hash
  hash256 hash;
  // 其他方法略...
}
\end{numberedc}

\section{上层数据结构}
\label{sec:upper-data-sec}

这个章节我们介绍上层数据结构,这些结构(类)会操作并处理下层的数据结构如\cola{账
  户Account}以及\colb{交易Transaction}.

\subsection{交易执行器接口 \texttt{ITxExecutable}}
\label{sec:ITxExecutable}

如果执行模块只暴露一个接口,那么将会是什么呢? 没错,那必然就是\cola{执行交
  易\texttt{executeTx()}}。那么既然是执行交易,就一定会访问下面的\cola{世界存
  储}。更准确的说是一个\Cola{只可以读的状态存储(readonly state storage)}。那么再结合之前
\cref{sec:general-world-storage}里提到过的:
\begin{center}
  \cola{世界状态是\textbf{地址address}到\textbf{账户account}的KV表}
\end{center}
我们可以得出世界状态的接口,见\cref{code:IAcnGettable}
\begin{numberedc}{只读的世界状态接口\texttt{IAcnGettable} (这个接口是专门给交易执行的)}{\label{code:IAcnGettable}}
  class IAcnGettable{
  public:
    virtual optional<Acn> getAcn(evmc::address addr) const noexcept=0;
  };\end{numberedc}
举个例子,这个接口可以这么使用:
\begin{simplec}
Tx t = ...;                     // 交易
IAcnGettable* w = ...;          // 执行器
evmc::address a = evmc::address(static_cast<uint64_t>(123));
optional<Acn> r = w->getAcn(a);          // 获取address为123的Account
if (r):                                  // 读到了
  Acn acn = r.value();                   // 获得Account
\end{simplec}
那么这样一来我们就可以定义\cola{可执行交易}的接口了,见\cref{code:ITxExecutable}。:
\begin{numberedc}{可执行交易的接口\texttt{ITxExecutable}}{\label{code:ITxExecutable}}
class ITxExecutable {
public:
  /**
   * @brief Probably the most important function that the execution module expose.
   *
   * If the execution is successful, a series of state-changes (called
   * "journal") and the execution-result is returned.
   *
   * @param w The read-only stateDb (world state).
   * @param t The transaction to be executed.
   * @return the pair (journal,result). {} if the execution fails.
   */
  virtual optional<tuple<vector<StateChange>,bytes>> executeTx(IAcnGettable * const w,
                                                               const Tx & t) const noexcept = 0;
};
\end{numberedc}
正如\cref{code:ITxExecutable}里的注释所说,如果交易执行成功的话将会返回一个对儿:
\begin{center}
  \cola{<日志 Journal>} , \colb{<交易执行结果 Result>}
\end{center}
类型为
\begin{center}
  \cola{\texttt{vector<StateChange>}} , \colb{\texttt{bytes}}
\end{center}
其中,\texttt{StateChange} 代表一个对KV表改变的操作,见\cref{code:StateChange}:
\begin{numberedc}{状态改变\texttt{StateChange}}{\label{code:StateChange}}
  /**
   * @brief A state change, should be generated by executor.
   */
  struct StateChange{bool del=false; string k; string v;};
\end{numberedc}
一组\cola{状态改变}会组成一个\cola{日志Journal},代表着一条交易执行后对\cola{世界状态}的改变。
比如说如果你手里有一个\texttt{ITxExecutable}则可以这么用:

\begin{simplec}
IAcnGettable * w = ...;         // 世界状态
Tx t = ...;                     // 交易
ITxExecutable * e = ...;        // 执行器

auto r = e->executeTx(w,t);
if (r){
  auto [journal , result] = r.value(); // 获取结果
}
\end{simplec}

因此这两个接口的关系是这样的(\cref{fig:exe-storage-relationship}):
\begin{figure}[h]
  \centering
  \begin{tikzpicture}
    % \draw[style=help lines] (-2cm,5cm) grid +(15cm,-10cm);
    \node (ITxExecutable) [text width=8cm] at (8cm,3cm)
    {\begin{ifaceBox}[title=\texttt{ITxExecutable}]
        \begin{simplec}
  optional<tuple<vector<StateChange>,bytes>>
             executeTx(IAcnGettable *w,Tx t);
        \end{simplec}
        \end{ifaceBox}};

      \node (IAcnGettable) [text width=7cm] 
      {\begin{ifaceBox}[title=\texttt{IAcnGettable}]
          \begin{simplec}
          optional<Acn> getAcn(evmc::address addr)
\end{simplec}
          \end{ifaceBox}};
        \Southtoeast{[xshift=-3cm]ITxExecutable}{IAcnGettable}
  \end{tikzpicture}
  \caption{执行模块接口和存储模块接口的依赖关系}
  \label{fig:exe-storage-relationship}
\end{figure}
\FloatBarrier                   % \usepackage{placeins}
\subsection{EVM交易执行器 \texttt{EvmExecutor}, 一个\texttt{ITxExecutable}的实现}
\label{sec:EvmExecutor}
\texttt{EvmExecutor} 是一个 \texttt{ITxExecutable} 的实现,用来执行EVM交易。这个
类里面包了一个EVM,不存储任何状态(stateless),因此一般这么使用:

\begin{simplec}
IAcnGettable *w = ...;
unique_ptr<ITxExecutable> e{new EvmExecutor()};
Tx t = ...;                     // 交易
auto o = e->executeTx(w,t);     // 执行EVM交易
if (o)
  auto [journal,res] = o.value(); // 获取值
\end{simplec}
因此我们就有了如下关系(\cref{fig:exe-storage-relationship2})
\begin{figure}[h]
  \centering
  \begin{tikzpicture}
    % \draw[style=help lines] (-2cm,5cm) grid +(15cm,-10cm);
    \node (ITxExecutable) [text width=8cm] at (8cm,3cm)
    {\begin{ifaceBox}[title=\texttt{ITxExecutable}]
        \begin{simplec}
  optional<tuple<vector<StateChange>,bytes>>
             executeTx(IAcnGettable *w,Tx t);
        \end{simplec}
      \end{ifaceBox}};
    \node (IAcnGettable) [text width=7cm] 
    {\begin{ifaceBox}[title=\texttt{IAcnGettable}]
        \begin{simplec}
          optional<Acn> getAcn(evmc::address addr)
\end{simplec}
\end{ifaceBox}};

\node (EvmExecutor) [text width=8cm] at (8cm,-3cm)
{\begin{varBox}[title=\texttt{EvmExecutor}]
     一个EVM交易执行器的实现
    \end{varBox}};

\Southtoeast{[xshift=-3cm]ITxExecutable}{IAcnGettable}
\draw[very thick, -latex] (EvmExecutor) -- (ITxExecutable);
\end{tikzpicture}
\caption{执行模块接口和存储模块接口,EVM执行器的依赖关系}
\label{fig:exe-storage-relationship2}
\end{figure}


\subsection{可读写世界状态存储的接口 \texttt{IWorldChainStateSettable}}
\label{sec:IWorldChainStateSettable}

在每笔\cola{交易}被执行过后,我们可以获得改变世界状态
的\cola{日志Journal}和\colb{结果Result},这些终究需要被写进\colc{世界存储World
  Storage}之中。而这就需要存储暴露一个\textbf{可以读写}的接口。

回忆我们在\cref{sec:general-world-storage}里面就有说过,区块链数据是\cola{账本}
+ \colb{状态},而背后对应的则是两个DB:
\begin{enumerate}
\item \cola{\texttt{chainDB}} :  对应 \cola{账本},这个DB只能添加,不能删除。
\item \colb{\texttt{stateDB}} :  对应 \colb{状态},这个可以被日志journal所改变。
\end{enumerate}

因此我们要求的接口自然就变成了:

\begin{enumerate}
\item \texttt{get/set} \texttt{chainDB}: 往\cola{\texttt{chainDB}}里“存”和“读”的
  基本的KV操作。
\item \texttt{applyJournal}: 往\colb{\texttt{stateDB}}里用一系列日志Journal。(把
  日志里对应的KV该删的删,该加的加)。
\end{enumerate}

于是这里我们可以来介绍我们的\cola{可读写世界状态存储的接
  口 \texttt{IWorldChainStateSettable}} 了。(\cref{code:IWorldChainStateSettable})

\begin{numberedc}{可读写世界状态存储的接口\texttt{IWorldChainStateSettable}}
{\label{code:IWorldChainStateSettable}}
  class IWorldChainStateSettable {
  public:
    virtual bool setInChainDB(const string k, const string v) =0; // 往chainDB里存
    virtual optional<string> getFromChainDB(const string k) const =0;// 从chainDB里读
    virtual bool applyJournalStateDB(const vector<StateChange> & j)=0;// 对stateDB apply日志
  };
\end{numberedc}

比如说,如果想要对\texttt{stateDB}进行改变的话一般是这样的:
\begin{simplec}
IWorldChainStateSettable * p = ...;

vector<StateChange> j = {
  {false, "k1", "v1"},          // 添加(k1,v1)
  {false, "k2", "v2"},          // 添加(k2,v2)
  {true, "k3", ""},             // 删(k3)
};

p->applyJournalStateDB(j);      // apply
\end{simplec}

而我们也看到了,\texttt{chainDB}就是个\texttt{string-string}的KV,因此可以想存什么就存什么,比如:
\begin{simplec}
IWorldChainStateSettable* p = ...;

bool ok = p->setInChainDB("/blockNumber","123"); // 存入
auto r = p->getFromChainDB("/blockNumber");
if(r)                           // 读到了
  string v = r.value(); // 读出
\end{simplec}

\subsection{世界存储 \texttt{WorldStorage} 实例}
\label{sec:world-storage}

终于,我们这里介绍真正的世界存储:\texttt{WorldStorage},一个:
\begin{itemize}
\item 符合以前要求 的\texttt{IAcnGettable}(\cref{sec:ITxExecutable})以及
\item \texttt{IWorldChainStateSettable} (\cref{sec:IWorldChainStateSettable}) 接口要求
\end{itemize}
的实例: 一个真正的(暂时)\cola{区块链存储World Storage}: \texttt{WorldStorage}
(\cref{code:WorldStorage})

\begin{numberedc}{世界存储}{\label{code:WorldStorage}}
class WorldStorage: public virtual IWorldChainStateSettable,
                      public virtual IAcnGettable
  {
  public:
    rocksdb::DB* chainDB;
    rocksdb::DB* stateDB;
  // ...
}
\end{numberedc}

可以看到这个类底下的实现是两个RocksDB, 而构建函数是会接受一个文件夹路经,之后在这
个路径里会生成两个文件夹\texttt{chainDB}还有\texttt{stateDB},并分别储存这两
个RocksDB的数据。
\begin{simplec}
  WorldStorage w{};               // 在当前目录存储

  // #include<filesystem>
  namespace filesystem =  std::filesystem;
  WorldStorage w{filesystem::temp_directory_path()}; // 在tmp目录存储

  IAcnGettable* p1 = dynamic_cast<IAcnGettable*>(&w); // 给执行接口用
  IWorldChainStateSettable* p2 = dynamic_cast<IWorldChainStateSettable*>(&w); // 可读写接口
\end{simplec}
因此最后我们执行存储模块的关系图就变成了如\cref{fig:exe-storage-relationship3}:

\begin{landscape}
 \begin{figure}[h]
  \centering
  \begin{tikzpicture}
    % \draw[style=help lines] (-2cm,5cm) grid +(15cm,-10cm);
    \node (ITxExecutable) [text width=8cm] at (8cm,3cm)
    {\begin{ifaceBox}[title=\texttt{ITxExecutable}]
        \begin{simplec}
          optional<tuple<vector<StateChange>,bytes>>
          executeTx(IAcnGettable *w,Tx t);
        \end{simplec}
      \end{ifaceBox}};
    \node (IAcnGettable) [text width=7cm] 
    {\begin{ifaceBox}[title=\texttt{IAcnGettable}]
        \begin{simplec}
          optional<Acn> getAcn(evmc::address addr)
        \end{simplec}
      \end{ifaceBox}};

    
    \node (IWorldChainStateSettable) [text width=8.5cm]  at (-3cm,8cm)
    {\begin{ifaceBox}[title=\texttt{IWorldChainStateSettable}]
        \begin{simplec}
          bool setInChainDB(string k,string v)
          optional<string> getFromChainDB(string k)
          bool applyJournalStateDB(vector<StateChange> j)
        \end{simplec}
      \end{ifaceBox}};

    \node (EvmExecutor) [text width=8cm] at (8cm,-3cm)
    {\begin{varBox}[title=\texttt{EvmExecutor}]
        一个EVM交易执行器的实现
      \end{varBox}};

    \node (WorldStorage) [text width=8cm] at (-8cm,4cm)
    {\begin{varBox}[title=\texttt{WorldStorage}]
        基于RocksDB的存储
      \end{varBox}};

    \Southtoeast{[xshift=-3cm]ITxExecutable}{IAcnGettable}
    % \easttosouth[o-]{IAcnGettable}{[xshift=-3cm]ITxExecutable}

    \Southtowest[-latex]{[xshift=3cm]WorldStorage}{IAcnGettable}
    \Northtowest[-latex]{WorldStorage}{IWorldChainStateSettable}
    \draw[very thick, -latex] (EvmExecutor) -- (ITxExecutable);
  \end{tikzpicture}
  \caption{执行模块和存储模块对内和对外的接口。上层模块一般调
    用\texttt{IWorldChainStateSettable}以及\texttt{ITxExecutable}}
  \label{fig:exe-storage-relationship3}
\end{figure}
\end{landscape}


\section{区块层数据结构}
\label{sec:Block-Exec}

\cref{sec:upper-data-sec}里描述并最后提供的接口是一个很独立,且设计就是要可以被很
容易地接入其他系统里的。这里我们就介绍一个基于上面所说的一个区块执行系统。

\subsection{区块 Block}
\label{sec:Blk}


所谓区块链肯定是要有\cola{区块Block},每个区块包含如下信息:
\begin{itemize}
\item 区块号 \texttt{number}
\item 区块哈希 \texttt{hash}
\item 上一个区块的哈希 \texttt{parentHash}
\item 区块里包含的交易 Transactions \texttt{txs}
\end{itemize}
因此我们有如下定义(\cref{code:Blk}):
\begin{numberedc}{区块Block 数据结构}{\label{code:Blk}}
class Blk: virtual public IJsonizable
         , virtual public ISerializable {
public:
  Blk() = default;
  Blk(const uint64_t n,const hash256 p,vector<Tx> t):
    number(n),
    txs(t),
    parentHash(p){/*构建区块,计算哈希*/};
  uint64_t number;
  hash256 hash;
  hash256 parentHash; /// empty for genesis block
  vector<Tx> txs;
  /*其他方法略了*/
}
\end{numberedc}

\subsection{已执行区块 Executed Block}
\label{sec:ExecBlk}

每个区块在被执行过后会新获得这些信息:
\begin{itemize}
\item 每条区块内交易所导致的\colb{状态改变State Changes}。
\item 每条区块内交易所对应的\cola{交易回执Transaction Receipts}。
\end{itemize}

这些都和\cref{sec:ITxExecutable}里所描述的执行结果对应。
具体的数据定义为:
\begin{numberedc}{已执行区块  Executed Block}{\label{code:ExecBlk}}
class ExecBlk: virtual public IJsonizable
             , virtual public ISerializable
             , public Blk{
public:
  vector<vector<StateChange>> stateChanges;
  vector<TxReceipt> txReceipts;
  ExecBlk() = default;
  ExecBlk(Blk b,
          vector<vector<StateChange>> j,
          vector<TxReceipt> r): Blk(b)
                              ,stateChanges(j)
                              ,txReceipts(r){
    // ...
  }
};
\end{numberedc}

所以区块和已执行的区块大概关系见\cref{fig:execBlk}。
\begin{figure}[h]
  \centering
  \begin{tikzpicture}
    % \draw[style=help lines] (-2cm,5cm) grid +(15cm,-10cm);

    \tikzstyle{sField} = [fill=\mycola!70,draw,ellipse,text=white]
    \tikzstyle{sTx} = [draw,rectangle,text width=2cm,text centered,rounded corners]

    \begin{scope}
      \matrix [minimum size=2em,row sep=0.5cm,column sep=1cm,fill=\mycola!20,
      inner sep=1cm,name=execBlk,rounded corners]{
        \node[sTx,name=t1] {交易1}; & \node[sField,name=r1] {结果1};\\
        & \node[sField,name=j1] {日志1};\\
        \node[sTx,name=t2] {交易2}; & \node[sField,name=r2] {结果2};\\
        & \node[sField,name=j2] {日志2};\\
        \node[sTx,name=t3] {交易3}; & \node[sField,name=r3] {结果3};\\
        & \node[sField,name=j3] {日志3};\\
      };

      \foreach \i in {1,2,3}{
        \easttowest[-o,thick,gray]{t\i}{j\i}
        \easttowest[-o,thick,gray]{t\i}{r\i}
      }
      \node[above right,yshift=2pt] at (execBlk.north west) {\textbf{已执行区块} \texttt{ExecBlk}};
      \node[above right,yshift=2pt] at (execBlk.south west) {\texttt{<number>,<hash>,<parentHash>}};
    \end{scope}

    \begin{scope}[xshift=8cm]
      \matrix [minimum size=2em,row sep=1cm,column sep=1.5cm,fill=\mycola!15,
      inner sep=1cm,name=Blk,rounded corners]{
        \node[sTx,name=t1] {交易1}; & \\ 
        \node[sTx,name=t2] {交易2}; & \\
        \node[sTx,name=t3] {交易3}; & \\
      };

      \node[above right,yshift=2pt] at (Blk.north west) {\textbf{区块} \texttt{Blk}};
      \node[above right,yshift=2pt] at (Blk.south west) {\texttt{<number>,<hash>,<parentHash>}};
    \end{scope}
    
  \end{tikzpicture}
  \caption{区块和已执行区块}
\label{fig:execBlk}
\end{figure}

\section{区块执行接口}
\label{sec:IBlkExecutable}

就像和交易执行接口一样(\cref{sec:ITxExecutable})这里我们同样有区块执行接口
\begin{numberedc}{IBlkExecutable}{\label{code:IBlkExecutable}}
  class IBlkExecutable {
  public:
    virtual ExecBlk executeBlk(const Blk & b) const noexcept = 0;
    virtual bool commitBlk(const ExecBlk & b) noexcept = 0;
  };
\end{numberedc}
其中\texttt{executeBlk()}会执行一个区块,第二个(语义上)会把执行过的区块写入存储。
而真正符合这个接口的实例则是\texttt{BlkExecutor}(\cref{code:BlkExecutor}):

\begin{numberedc}{BlkExecutor}{\label{code:BlkExecutor}}
class BlkExecutor: public virtual IBlkExecutable{
public:
  IWorldChainStateSettable* const world;
  ITxExecutable* const txExecutor;
  IAcnGettable* const readOnlyWorld;
  BlkExecutor(IWorldChainStateSettable* const w,
              ITxExecutable* const e,
              IAcnGettable* const r
              ): world(w),
                 txExecutor(e),
                 readOnlyWorld(r){
    // ...
  };
  // 具体实现...
}
\end{numberedc}

可以看到这个\texttt{BlkExecutor}会接受之前在\cref{sec:upper-data-sec}之中定义的三个接口。(见\ref{fig:exe-storage-relationship4})

\begin{landscape}
 \begin{figure}[h]
  \centering
  \begin{tikzpicture}
    % \draw[style=help lines] (-2cm,5cm) grid +(15cm,-10cm);
    \node (ITxExecutable) [text width=8cm] at (8cm,3cm)
    {\begin{ifaceBox}[title=\texttt{ITxExecutable}]
        \begin{simplec}
          optional<tuple<vector<StateChange>,bytes>>
          executeTx(IAcnGettable *w,Tx t);
        \end{simplec}
      \end{ifaceBox}};
    \node (IAcnGettable) [text width=7cm] 
    {\begin{ifaceBox}[title=\texttt{IAcnGettable}]
        \begin{simplec}
          optional<Acn> getAcn(evmc::address addr)
        \end{simplec}
      \end{ifaceBox}};

    
    \node (IWorldChainStateSettable) [text width=8.5cm]  at (-3cm,8cm)
    {\begin{ifaceBox}[title=\texttt{IWorldChainStateSettable}]
        \begin{simplec}
          bool setInChainDB(string k,string v)
          optional<string> getFromChainDB(string k)
          bool applyJournalStateDB(vector<StateChange> j)
        \end{simplec}
      \end{ifaceBox}};

    \node (EvmExecutor) [text width=8cm] at (8cm,-3cm)
    {\begin{varBox}[title=\texttt{EvmExecutor}]
        一个EVM交易执行器的实现
      \end{varBox}};

    \node (WorldStorage) [text width=8cm] at (-8cm,4cm)
    {\begin{varBox}[title=\texttt{WorldStorage}]
        基于RocksDB的存储
      \end{varBox}};

    \node (BlkExecutor) [text width=8cm] at (8cm,6cm)
    {\begin{varBox}[title=\texttt{BlkExecutor}]
        区块执行器
      \end{varBox}};

    \node (IBlkExecutable) [text width=8.5cm]  at (8cm,9cm)
    {\begin{ifaceBox}[title=\texttt{IBlkExecutable}]
        \begin{simplec}
    ExecBlk executeBlk(Blk & b)
    bool commitBlk(ExecBlk & b)\end{simplec}
      \end{ifaceBox}};

    \easttowest[o-]{IWorldChainStateSettable}{BlkExecutor}
    \southtonorth[latex-]{IBlkExecutable}{BlkExecutor}
    \westTonorth[-o]{BlkExecutor}{IAcnGettable}
    % \Northtowest[o-]{IAcnGettable}{BlkExecutor}
    \westTowest[-o]{BlkExecutor}{ITxExecutable}

    \Southtoeast{[xshift=-3cm]ITxExecutable}{IAcnGettable}
    % \easttosouth[o-]{IAcnGettable}{[xshift=-3cm]ITxExecutable}
    \Southtowest[-latex]{[xshift=3cm]WorldStorage}{IAcnGettable}

    \Northtowest[-latex]{WorldStorage}{IWorldChainStateSettable}
    \draw[very thick, -latex] (EvmExecutor) -- (ITxExecutable);
  \end{tikzpicture}
  \caption{区块执行器\texttt{IBlkExecutable}所使用和调用的接口}
  \label{fig:exe-storage-relationship4}
\end{figure}
\end{landscape}
% Local Variables:
% TeX-engine: luatex
% TeX-command-extra-options: "-shell-escape"
% TeX-master: "m.tex"
% End: