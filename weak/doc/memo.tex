\documentclass[dvipsnames]{article}
% \documentclass[dvipsnames]{ctexart}
\title{weak-chain memo}



\usepackage{geometry}\geometry{
  a4paper,
  total={170mm,257mm},
  left=20mm,
  top=20mm,
}

\usepackage{tabularray}
\UseTblrLibrary{booktabs,siunitx}

\usepackage{adjustbox}          %to narrower the caption
\newlength\mylength
% \usepackage{hyperref}
\usepackage{svg}
\usepackage[skip=5pt plus1pt, indent=0pt]{parskip}
\usepackage{emoji}
% \setemojifont{NotoColorEmoji.ttf}[Path=C:/Users/congj/repo/myFonts/]
% \setemojifont{TwitterColorEmoji-SVGinOT.ttf}[Path=C:/Users/congj/repo/myFonts/]

\usepackage{booktabs}

\usepackage{tikz}
\usetikzlibrary{shapes} % ellipse node shape
\usetikzlibrary{shapes.multipart} % for line breaks in node text
\usetikzlibrary{arrows.meta}    %-o arrow head
\usetikzlibrary{arrows}
\usetikzlibrary{matrix}

\usepackage{amsmath}

\usepackage{minted}
\usepackage{tcolorbox}
\tcbuselibrary{skins}
\tcbuselibrary{minted}

\usepackage{fontspec}
\setmonofont{Cascadia}[
Path=/usr/share/fonts/truetype/Cascadia_Code/,
Scale=0.85,
Extension = .ttf,
UprightFont=*Code,              %find CascadiaCode.ttf
BoldFont=*CodePL,               %find CascadiaCodePL.ttf ...
ItalicFont=*CodeItalic,
BoldItalicFont=*CodePLItalic
]

% --------------------------------------------------
% Windows
% \setmonofont{Cascadia}[
% Path=C:/Windows/Fonts/,
% Extension = .ttf,
% UprightFont=*Mono,              %find CascadiaMono.ttf
% BoldFont=*Code,               %find CascadiaCodePL.ttf ...
% ItalicFont=*Code,
% BoldItalicFont=*Code
% ]


% Without this, there's a gap between the upper bbox and the node box

\newtcblisting{simplec}{
  listing engine=minted,
  minted language=c++,
  minted style=vs,
  minted options={fontsize=\small,autogobble,
  % framesep=1cm
  },
  tile,
  listing only,
  % bottom=0cm,
  % nobeforeafter, 
  boxsep=0mm,
  left=1mm,
  opacityback=0.5,
  colback=gray!20
}

\newcounter{myCListing}
\newtcblisting{numberedc}[3][]{
  listing engine=minted,
  minted language=c++,
  minted style=vs,
  minted options={fontsize=\small,autogobble,
  % framesep=1cm
  },
  tile,
  listing only,
  % bottom=0cm,
  % nobeforeafter, 
  boxsep=1.5mm,
  left=1mm,
  opacityback=0.5,
  colback=gray!20,
  phantom={
  \refstepcounter{myCListing}
  #3
  },
  title={代码~\themyCListing: #2},
  #1
}


\newtcblisting{simplepy}{
  listing engine=minted,
  minted language=python,
  minted style=vs,
  minted options={fontsize=\small,autogobble},
  tile,
  listing only,
  % nobeforeafter, 
  boxsep=0.5mm,
  left=1mm,
  opacityback=0.5,
  colback=gray!20,
}


\tcbset{enhanced, fontupper=\small,
  % show bounding box
}

% Color
\newcommand{\mycola}{MidnightBlue}
\newcommand{\mycolb}{Mahogany}
\newcommand{\mycolc}{OliveGreen}

\newcommand{\cola}[1]{\textcolor{\mycola}{#1}}
\newcommand{\colb}[1]{\textcolor{\mycolb}{#1}}
\newcommand{\colc}[1]{\textcolor{\mycolc}{#1}}
\newcommand{\Cola}[1]{\textcolor{\mycola}{\emph{#1}}}


\newcommand{\myTwo}[2]{\texttt{#1}为\texttt{#2}}

\tcbset{right=1mm,halign=flush left}
\newtcolorbox{ifaceBox}[1][]{colframe=\mycola, nobeforeafter, 
  fontupper=\footnotesize, left=2mm, #1}

\newtcolorbox{varBox}[1][]{nobeforeafter, 
  fontupper=\footnotesize, left=2mm, #1}

\newtcolorbox{weakBox}[1][]{colframe=gray!80, nobeforeafter,
  fontupper=\footnotesize, left=2mm, #1}

\newtcolorbox{funcBox}[1][]{colframe=\mycolb, nobeforeafter, 
  fontupper=\footnotesize, left=2mm, #1}

\newtcolorbox{noteBox}[1][]{colframe=\mycolc, nobeforeafter, 
  fontupper=\footnotesize, left=2mm, #1}

\newtcbox{\libbox}[1][\mycala]{on line,
  arc=0pt,outer arc=0pt,colback=#1!10!white,colframe=#1!50!black,
  boxsep=0pt,left=1pt,right=1pt,top=2pt,bottom=2pt,
  boxrule=0pt,bottomrule=1pt,toprule=1pt}

% copied from tcolorbox mannual
\newtcbox{\mylib}[1][\mycola]{on line,
  arc=0pt,outer arc=0pt,colback=#1!10!white,colframe=#1!50!black,
  boxsep=0pt,left=1pt,right=1pt,top=2pt,bottom=2pt,
  boxrule=0pt,bottomrule=1pt,toprule=1pt,
  left skip=0.1cm,
  right skip=0.1cm,
  fontupper=\ttfamily
}


% Redefine em
%latex.sty just do: \DeclareTextFontCommand{\emph}{\em}

\let\emph\relax % there's no \RedeclareTextFontCommand
\DeclareTextFontCommand{\emph}{\bfseries\em}

% Redefine section
\usepackage[small]{titlesec}


% curve version
% \newcommand\uptoleft[2][]{\draw[very thick,->](#1.south) to [out=270,in=180] (#2.west);}

% right-angle version
\newcommand\Southtowest[3][-o]{\draw[very thick,#1](#2.south) |- (#3.west);}
\newcommand\Southtoeast[3][-o]{\draw[very thick,#1](#2.south) |- (#3.east);}
\newcommand\Northtowest[3][-o]{\draw[very thick,#1](#2.north) |- (#3.west);}
\newcommand\Northtoeast[3][-o]{\draw[very thick,#1](#2.north) |- (#3.east);}

\newcommand\southtonorth[3][-o]{\draw[very thick,#1](#2.south) to [out=270,in=90] (#3.north);}
\newcommand\northtosouth[3][-latex]{\draw[very thick,#1](#2.north) to [out=90,in=270] (#3.south);}


\newcommand\easttowest[3][-latex]{\draw[very thick,#1](#2.east) to[out=0,in=180] (#3.west);}
\newcommand\easttonorth[3][-latex]{\draw[very thick,#1](#2.east) to[out=0,in=90] (#3.north);}
\newcommand\easttosouth[3][-latex]{\draw[very thick,#1](#2.east) to[out=0,in=270] (#3.south);}

% the left node link my ``up''

\newcommand\westTowest[3][-latex]{\draw[very thick,#1](#2.west) to[out=180,in=180] (#3.west);}
\newcommand\westTonorth[3][-latex]{\draw[very thick,#1](#2.west) to[out=180,in=90] (#3.north);}
\newcommand{\mySection}[1]{\section*{\texttt{#1}}}

\tikzstyle{every node}=[inner sep=0pt]
\tikzstyle{iface}=[rectangle,fill=gray!30,inner sep=2mm]
\tikzstyle{myMatrix}=[matrix of nodes,below right,
nodes={right},                  %apply to all nodes
row sep=1cm,column sep=2cm]

% TREE NODE
\tikzstyle{myTreeNode}=[draw=\mycola,anchor=west,inner sep=2mm,rectangle,rounded corners,fill=gray!20]
\tikzstyle{edge from parent}=[draw=\mycola,thick,-latex]
\tikzstyle{every child node}=[myTreeNode]

\usepackage{simpsons}
\usepackage{cleveref}
\crefname{figure}{图}{图}
%                    ^^^ plural
\Crefname{figure}{图}{图}
%         ^^^^^^ type = counter name

\crefname{table}{表}{表}
\Crefname{table}{表}{表}

\crefname{section}{}{}
\Crefname{section}{}{}
\creflabelformat{section}{第#2#1#3章节}


% #2 : start of hyperlink
% #3 : end of hyperlink
% #1 : The counter

% U need to define both crefname and Crefname , else it will cause fatal error.
\crefname{myCListing}{代码}{代码}
\Crefname{myCListing}{代码}{代码}
\date{\today}
\author{HiAaa}
\tcbuselibrary{breakable}
\newtcolorbox{blackbox}{tile,colback=black,colupper=white,nobeforeafter,halign=flush center}

\newcommand{\mycolaa}{\mycola!20}

\usepackage{placeins}

% \usepackage{lscape}             %for landscape env
\usepackage{pdflscape} %uncomment this  and comment above line to see the difference
% --------------------------------------------------

\usepackage{tabularx}



\begin{document}
\maketitle
\tableofcontents{}
\newpage{}

% 
\section*{2024-01-23}

\subsection*{serious tx verification @ \texttt{class Tx} \cola{\texttt{core.hpp}}}
\emoji{parrot} : I just feels like it would make sense to include the methods
about verifying Tx within the class \text{Tx} itself? Kinda OOP paradigm right?

\emoji{turtle} : What you have in mind?

\emoji{parrot} : I think the simplest way is just to define:

\begin{simplec}
bool verify() const;
bool verify(caKey k) const;
\end{simplec}

The first one is used when we are using \texttt{--serious-tx-check
  \{1|public-chain-mode\}}, in which we check the wether the \texttt{pk\_pem}
and \texttt{signature} match (according to the contents).

The second one is used when we are using \texttt{--serious-tx-check
  \{2|private-chain-mode\}}, in which, in addition to what we check above, we
also check the \texttt{pk\_crt} field to see if it is signed by the CA \texttt{k}.

\emoji{turtle} : Sounds alright. So where do all those SSL stuff come from ? I
remember currently \texttt{core.hpp} is not including any SSL related headers.
And in order to reduce the footprint of OpenSSL, we have put all those crypto
stuff in \text{class SslMsgMgr} in \texttt{net/pure-netAsstn.hpp}. Perhaps most
importantly it contains a couple of static helper function (which is pretty much
all we learnt from the OpenSSL tutorial) such as:
\begin{simplec}
static optional<UniquePtr<EVP_PKEY>> load_key_from_pem(const std::string& s, bool is_secret = true)
static bool do_verify(EVP_PKEY *ed_key,const string msg, const string sig)
\end{simplec}

\emoji{parrot} : Oh.. I feels like these two functions are just what we need
right? I mean, it feels like all we need is:

\begin{simplec}
// var: ca_key, tx
do_verify(ca_key, tx.pk_pem, tx.pk_crt);

UniquePtr<EVP_PKEY> pk = load_key_from_pem(tx.pk_pem, false).value(); // false means public key
string msg = tx.nonce + tx.data; // this getter can be implemented in Tx
do_verify(pk.get(), msg, tx.signature);
\end{simplec}

\emoji{turtle} : Great, it seems like the only two things we need are these two
static methods right? Yeah. Let's do that.

\subsection*{12:12 serious nonce check}

\colz{
  \emoji{parrot} : Ohhh, oh, also, I think the \texttt{SeriousDivExecutor} should
  also check that wether each \texttt{Tx} has a unique \texttt{nonce}. More
  specifically, the pair \texttt{<from, nonce>} should be unique. This is to
  prevent the \texttt{from} from double spending.

  \emoji{turtle} : I remember we did some check on this in the \texttt{Mempool}
  (The Tx pool) right? Ohh, we checked the uniqueness of the \texttt{Tx.hash}.

  \emoji{parrot} : Ohh..checking hash is better, because it depends on the whole
  Tx including the \texttt{nonce} and \texttt{from}... Yeah, we were smarter back
  then....
}

\emoji{turtle} : So, yeah, we should check the uniqueness of the
\texttt{Tx.hash}. Emm. this can be done by passing the argument (obtained from
\texttt{init.hpp}) of type \texttt{std::unordered\_set<hash256>;}, just like the
way \texttt{Mempool} does.

\emoji{parrot} : Let's do that. Wait, wait , wait... Why I don't see how the
hash is calculated in \texttt{Tx}?

\emoji{turtle} : Oh, it seems like the logic of calculating hash is within the ctor:

\begin{simplec}
Tx(const address f,const address t,const bytes d,const uint64_t n, Type type = Type::evm)
\end{simplec}

And the Rpc will call a special factory method to create a \texttt{Tx} object

\begin{simplec}
optional<tuple<string,vector<Tx>>> parse_txs_jsonString_for_rpc(string_view s)
\end{simplec}

Em... I feels like we should move the hash calculation logic to a separate
method....like:

\begin{simplec}
hash256 getHash() const;
\end{simplec}

\emoji{parrot} : Yeah, I think that's better. 

\emoji{turtle} : I feels like there're some duplication of code between the
\texttt{parse\_txs\_json\_for\_rpc(json::array \&\& a)} and
\texttt{fromJson(const json::value \&v)}. Let's merge it.

\emoji{parrot} : Yeah, but I think it's more like a refactoring thing, because
the method called by the Rpc is required to throw an informative exception, but
\texttt{bool fromJson()} is marked as \texttt{noexcept} and uses its return
value to indicate the success or failure of parsing.

\emoji{turtle} : So we can refactor into something like:

\begin{simplec}
bool fromJson(const json::value &v) noexcept override {
  try {
    fromJson0(v);
  }catch (const std::exception &e){
    return false;
  }
}

void fromJson0(const json::value &v) {
  // here we can throw exceptions
}
\end{simplec}

\emoji{parrot} : Ohh, I just remembered that why we seperated the two
(\texttt{fromJson} and \texttt{parse\_txs\_json\_for\_rpc(json::array \&\& a)})... It's
because that \texttt{fromJson} is supposed to be called to those string
retrieved from the local db, which is supposed to be correct. But the latter is
supposed to be called on user-supplied string, which can be anything....

\emoji{turtle} : Oh, yeah. But I still think that we should refactor it. I think
skipping error check in \texttt{fromJson} is not a good idea...

\emoji{parrot} : Agree, let's do that.
% 
\section*{2024-01-30}

\subsection*{10:10 diable \texttt{command\_history} \cola{[done at 11:26]}}

\emoji{parrot} : Let's add an option for the \texttt{ListenToOneConsensus} to
disable the \texttt{command\_history}.

\emoji{turtle} : Let's go. Then I think let's just make it a new consensus, what
about \texttt{Solo-static}? That means no nodes can be dynamically added (after
the some blks are made.)

\subsection*{11:27 pb more}

\emoji{parrot} : I think now it's time to protobuf more. Such as \texttt{Tx} and
\texttt{Blk}.

\emoji{turtle} : Em... Yeah, let's do that, first, I think \texttt{Tx, Acn,
  Blk}. But note that things are bit more complicated with \texttt{Blk}, because
it inherits from \texttt{BlkHeader}, and also.. Oh, why it that?

\emoji{parrot} : I remember it's for historical reason. Previously, we have two
types of \texttt{Blk}, one is \texttt{Blk} which contains the Tx payloads, and
another \texttt{BlkForConsensus} which only contains the hashes of \texttt{Tx}s.
This is used when we have two kinds of consensus message: \texttt{AddTxs} and
\texttt{ExecBlk}. However, since we have been using \texttt{light-exe} lately
which doesn't need \texttt{BlkForConsensus}, we have almost forgot about it....

\emoji{turtle} : Yeah, let's still pb it. Also the \texttt{ExecBlk}, which is
the executed Blk.

\emoji{parrot} : Emm. Also, I think \texttt{WITH\_PROTOBUF} directive is not
neccesary, let's just make pb a requirement.

\emoji{turtle} : A sec. I think we should still keep the
\texttt{WITH\_PROTOBUF}, because sometimes, we do wanna use json serialization,
for example, when we are debugging. So let's just change the ``meaning'' of it.
The protobuf is always required, but when we are \texttt{WITH\_PROTOBUF}, we
serialize things with pb (as much as possible), otherwise, we use json.

\emoji{parrot} : Yeah, key-word reuse, I like it. (kinda C-ish)

\subsection*{18:11 start testing Blk pb}
\emoji{turtle} : How's your day?

\emoji{parrot} : I just finished and tested the \texttt{Tx} pb (in the simplest form, no
crypto stuff such as \texttt{pk\_pem} is involved). And I am about to test the
\texttt{Blk} pb.

\emoji{turtle} : I see...No rush.
% \section*{2024-02-11}

\subsection*{11:40 python as a vm}
\emoji{parrot}: Let's talk about python as a VM.

\emoji{turtle} : First let's see a minimal python contract:

\begin{simplepy}
from typing import Any

def hi() -> str:
    return "hi"

\end{simplepy}

\emoji{parrot} : Okay, I think there're two things we should do:

\begin{enumerate}
\item Check that the there's nothing dangerous.
\item Parse the function definition, which includes each function's name and the
  name of their arguments.
\end{enumerate}

\emoji{turtle}: Yeah. These two steps are separate, and we discuss them one by one. But before that,
let's talk about the contract's structure.

A contract is just a python file, but with some restrictions. The content of the
file is divided into two parts.

\begin{enumerate}
\item the \cola{header}, where all the \texttt{import}'s are
\item the \cola{body}, where all the function definitions are
\end{enumerate}

\begin{center}
  \begin{tikzpicture}
    % \draw[style=help lines] (0,0) grid +(10,-5);
    \node[draw,rectangle,below right] (A) {
      \begin{minipage}{10cm}
\begin{simplepy}
from typing import Any
# --------------------------------------------------
def hi() -> str:
    return "hi"
      \end{simplepy}
    \end{minipage}
  };
  % \node[draw,rectangle,fill=\mycolaa,right of=header,xshift=3cm] (body) {body};
  \draw[latex-latex] ([shift={(0.5cm,0)}]A.north east) -- +(0,-2em) coordinate
  (B) node[midway,right,xshift=1em] {the header};
  \draw[latex-latex] ([shift={(0.5cm,0)}]A.south east) -- (B)
  node[midway,right,xshift=1em] {the body};
\end{tikzpicture}
\end{center}

The start of a \texttt{def} marks the end of the header.

In the header area, only \texttt{from ... import ...} is allowed, but star
import is not allowed.

And the allowed symbols currently are:
\begin{center}
  \begin{tblr}{|l|c|}
    \hline
    Symbol & {Description} \\
    \hline
    \texttt{math.*} & the standard math module \\
    \texttt{cmath.*} & the standard complex math module \\
    \texttt{typing.*} & the module for type hints \\
    \texttt{hashlib.*} & the module for hashing \\
    \texttt{hmac.*} & the module for hash-based message authentication code \\
    \hline
  \end{tblr}
\end{center}

\emoji{parrot} : I see. So the first step is to check the header. We need to
make sure that there's no dangerous import. And next we check the rest (the
body), right?

\emoji{turtle} : Yeah. 


\subsubsection*{Validity check}
\emoji{turtle} : The first step is to check that the contract is safe to run. We
check two things:

\begin{enumerate}
\item The header is safe
\item The body is safe
\end{enumerate}

For the header, we just need to check those imports.

For the body, we need to check that there's no ``dangerous statement''. In
particular, we wanna make sure that

\begin{enumerate}
\item there's no \texttt{import} statement
\item Some built-in functions are not allowed.
\end{enumerate}

\emoji{parrot} : Okay, so what are the built-in functions that are not allowed?

\emoji{turtle} : The built-in functions can be found at
\url{https://docs.python.org/3/library/functions.html}. But if we need all
``symbols'' in the default default namespace, we can use (Copilot told me):

\begin{simplepy}
import builtins

# Get a list of all built-in functions and variables
builtin_functions_and_variables = dir(builtins)

# Print them out
for name in builtin_functions_and_variables:
    print(name)
\end{simplepy}

Currently, the following
are considered dangerous|inappropriate and therefore not allowed

\begin{center}
  \begin{tblr}{|l|c|}
    \hline
    Function & {Description} \\
    \hline
    % \texttt{print} &  print to the standard output \\
    \texttt{\_\_import\_\_} &  import a module \\
    \texttt{breakpoint} &  enter the debugger \\
    \texttt{compile} &  compile a string into a code object \\
    \texttt{eval} &  evaluate a string as a python expression \\
    \texttt{execfile} & execute a file \\
    \texttt{exec} &  execute a string as a python statement \\
    \texttt{get\_ipython} &  get the current IPython instance \\
    \texttt{globals} &  return the global symbol table \\
    \texttt{memoryview} & create a memoryview object \\
    \texttt{help} &  get help on an object \\
    \texttt{id} & get the identity of an object, usually the memory address.
    This should be removed because the result is not predictable. \\
    \texttt{input} &  read a line from the standard input \\
    \texttt{open} &  open a file \\
    \texttt{quit / exit} &  exit the interpreter \\
    \texttt{runfile} &  run a file \\
    \texttt{vars} & return the \texttt{\_\_dict\_\_} attribute of an object \\
    \hline
  \end{tblr}
\end{center}




\section*{2024-02-18}
\subsection*{19:03 the python contract verifier}

\emoji{parrot} : Yeah, after a journey to the \texttt{ast} standard library, I
think now we can have a try to write a contract verifier in python.

\emoji{turtle} : Yeah, so what are the rules?

\emoji{parrot} : What do you think ? Are our rules unchanged?

\emoji{turtle} : More or less.... But I think now we can make it more
specific (since we now know \texttt{ast}.):

\begin{enumerate}
\item At the top level, only \texttt{def},\texttt{import} and \texttt{from .. import}
  are allowed. In particular, only those allowed modules are allowed.(see
  2024-02-11).
\item The keywords listed in 2024-02-11 are banned. In particular, they cannot
  appear as function name, variable name, argument name... (Pretty much every
  where except for the comments and string literal.)
\end{enumerate}

\section*{2024-02-18}
\subsection*{continued: the python contract verifier}
\emoji{parrot} : Oh, so great. But I remember we need to write two tools: one
for syntax verifying and one for function argument parsing.

\emoji{turtle} : Yeah, \texttt{ast} can help us both.

\section*{2024-03-05}
\subsection*{start writing the function parser}
\emoji{parrot} : I think we can start with the function parser.

\emoji{turtle} : Yeah, so I think it's kinda like extracting the \cola{header}
from the script right?

\emoji{parrot} : Yeah, so I think a good start would be:

\begin{enumerate}
\item walk through the top level def
\item extract the function name, argument names. During this, we should be able
  to mark that if a function has required the \cola{magical argument}, for now,
  they are \texttt{\_tx\_context : dict[str,Any]} and \texttt{\_tx\_storage :
    dict[str,Any]}. These special args are supposed to be filled by the vm, and
  are not exposed to the abi. (So when a user invoke a method, they don't
  specify these arguments in their \cola{abi call})
\end{enumerate}

\emoji{turtle} : Why don't we just mark all the arguments that starts with
underscore as \cola{magical arguments}?

\emoji{parrot} : That's a good idea.

\emoji{turtle} : Yeah, so that we can add more \cola{magical arguments} later..

So given the contract
\begin{simplepy}
from typing import Any
from math import sqrt

def hi() -> str:
    return "hi"

def plus_one(x : int) -> int:
    return x + 1

def set(key: str, value: Any, _storage: dict[str,Any]) -> None:
    _storage[key] = value

def get(key: str, _storage: dict[str,Any]) -> Any:
    return _storage[key]

def init(_storage: dict[str,Any], _tx_contet : dict[str,Any]) -> None:
    _storage["tx_from"] = _tx_context["from"]  # address hex string
    _storage["tx_to"] = _tx_context["to"]  # address hex string
    _storage["tx_hash"] = _tx_context["hash"]  # hex string
    _storage["tx_timestamp"] = _tx_context["timestamp"]  # int
    #  : storage is a dictionary that should be serializable into JSON

## : So this should produce the abi object:
"""
a = {
    'hi' : {},
    'plus_one' : {'args': ['x']},
    'set' : {'args': ['key', 'value'], 'special_args' : ['_storage']},
    'get' : {'args': ['key'], 'special_args' : ['_storage']},
    'init' : {'special_args' : ['_storage', '_tx_context']}
}

# : and to invoke a method, use something like:

a = {
    'method' : 'plus_one',
    'args' : {'x' : 1}
}
"""
\end{simplepy}

\section*{2024-03-11}

\emoji{parrot} :  Okay, let's try writing a python vm executor.

\emoji{turtle} : Great, first, let's talk about the interface.

\emoji{parrot} : Let's say

\begin{simplec}
tuple<bool, string> execute_py_tx(string_view py_contract, string_view abi, int timeout_s = 2);
\end{simplec}

\emoji{turtle} : What's the return value?

\emoji{parrot} : I am thinking the first one is the success or not, the second
one is the error message, or the result. The second string is something should
be put into the \texttt{TxReceipt}.

\emoji{turtle} : What's would the return value be when things go well?

\emoji{parrot} : I think

\begin{simplejs}
{
    "result" : 1
    "log" : "Here is some log printed by the methods"
}
\end{simplejs}

The result can be any json object.

And when things go wrong, it would be
\begin{simplejs}
{
    "error" : "Here are some error messages"
}
\end{simplejs}

\emoji{turtle} : Oh, so the boolean return value can acutally be inferred from
the json object right?

\emoji{parrot} : Yes, but I think it's better to have a boolean return
value.(just for convenience)

\section*{2024-03-11}

\emoji{parrot} : Finally, let's try the pyExecuto

\end{document}

% Local Variables:
% TeX-engine: luatex
% TeX-command-extra-options: "-shell-escape"
% TeX-master: "m.tex"
% TeX-parse-self: t
% TeX-auto-save: t
% End: