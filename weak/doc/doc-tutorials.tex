\section{Tutorials}
\label{sec:tut}

\subsection{Start from the minimal}
\label{sec:minimal}
\emoji{parrot} : What's the minimal example of \Wch{} ?

\emoji{turtle} : You can start a node in \cola{Solo} mode. This is useful if you
want to familiarize yourself with the API, such as how to send transactions.
It's also good for testing smart contracts.

\begin{simplesh}
wch --port 7777 # start a node on port 7777 in Solo mode
\end{simplesh}

Now we have no blocks yet
\begin{simplesh}
curl http://localhost:7777/get_latest_Blk # []
\end{simplesh}

\subsubsection{Deploy an Evm contract}
To deploy an Evm contract, try:
\begin{simplesh}
txs='[
  {"from" : "01",
   "to" : "",
   "data" : "608060405234801561001057600080fd5b50610150806100206000396000f3fe608060405234801561001057600080fd5b50600436106100365760003560e01c806360fe47b11461003b5780636d4ce63c14610057575b600080fd5b610055600480360381019061005091906100c3565b610075565b005b61005f61007f565b60405161006c91906100ff565b60405180910390f35b8060008190555050565b60008054905090565b600080fd5b6000819050919050565b6100a08161008d565b81146100ab57600080fd5b50565b6000813590506100bd81610097565b92915050565b6000602082840312156100d9576100d8610088565b5b60006100e7848285016100ae565b91505092915050565b6100f98161008d565b82525050565b600060208201905061011460008301846100f0565b9291505056fea2646970667358221220271e30d641d99bedebb5450b18efe8b67269cf688a15386162d4c2ff7072a8af64736f6c63430008130033",
   "nonce" : 123
  }
]'
e=http://localhost
curl --data $txs $e:7777/add_txs
\end{simplesh}
The contract is compiled using
\begin{simplesh}
solc hi.sol --bin
\end{simplesh}
from a \texttt{hi.sol} file:
\begin{simplesol}
  // SPDX-License-Identifier: GPL-3.0
  pragma solidity >=0.4.16 <0.9.0;

  contract SimpleStorage {
    uint storedData;

    function set(uint x) public {
      storedData = x;
    }

    function get() public view returns (uint) {
      return storedData;
    }
  }
\end{simplesol}

This should return the transaction hash and the deployed address which is
something like
\begin{simplejs}
[
    {
        "hash": "b06b89b665df6f7ff1be967faf9a0601f71c0a3cdb8250c48fe7fcc663b18d1b",
        "deployed_address": "0000000000000000000000000000000064d05d1b"
    }
]
\end{simplejs}

\subsubsection{Invoke an Evm Contract}
Let's \texttt{set(123)} (invoke the method \texttt{set} with the argument 123).
The data is the Ethereum ABI encoding of the method signature and the argument.
\begin{simplesh}
$deployed_addr=0000000000000000000000000000000064d05d1b
txs="[{
  \"from\" : \"01\",
  \"to\" : \"$deployed_addr\",
   \"data\" : \"60fe47b1000000000000000000000000000000000000000000000000000000000000007b\",
   \"nonce\" : 125
  }]"
curl --data $txs $e:7778/add_txs
\end{simplesh}

Then \texttt{get()} (invoke the method \texttt{get}).
\begin{simplesh}
txs="[{
  \"from\" : \"01\",
  \"to\" : \"$deployed_addr\",
  \"data\" : \"6d4ce63c\",
  \"nonce\" : 126
}]"

curl --data $txs $e:7777/add_txs
# [{"hash":"f192381aa197b851870e98fc515414c51d2311fea7e5bc01c16dceb6a7fed3d9"}]
\end{simplesh}

The result of method invocation is stored in the \cola{transaction receipt}, we
can get it by
\begin{simplesh}
curl $e:7777/get_receipt?hash=f192381aa197b851870e98fc515414c51d2311fea7e5bc01c16dceb6a7fed3d9

# The result of get() is 0x7b = 123 as expected
# {"ok":true,"result":"000000000000000000000000000000000000000000000000000000000000007b"}
\end{simplesh}

You can retrieve the transaction by looking at the latest block
\begin{simplesh}
curl http://localhost:7777/get_latest_Blk # [{
...
"txs": [{...your transaction should be here...}]
}]
\end{simplesh}

\begin{weakBox}[title=Note]
  \emoji{turtle} : Currently \Wch{} does not provide a \texttt{getTxByHash()}
  API. This is because \Wch{} would have no better way than to scan the whole
  blockchain and may cause performance issues. So usually the user (or probably
  some agents) should try to record what transactions are in which block by keep
  tracking the latest block.

  All of these can by achieved by calling \texttt{curl
    http://localhost:7777/get\_latest\_Blk} to get the latest block or
  \texttt{curl http://localhost:7777/get\_latest\_Blk?n=3} to get the latest 3
  blocks.
\end{weakBox}

\subsection{The simplest transaction: Pure Data transaction}
\emoji{parrot} : What if I just want to store some data on the blockchain? For
example just a pdf. Is there a simple way to do that?

\emoji{turtle} : In that case, \Wch{} has \cola{pure data transaction}.

Suppose we have an \Wch{} endpoint at \texttt{http://localhost:7777} (as can be
done in \cref{sec:minimal}), we can send a pure data transaction by
\begin{simplesh}
txs="[{
  \"from\" : \"01\",
  \"to\" : \"\",
  \"type\" : \"data\",
  \"data\" : \"aaa\",
  \"nonce\" : 123
}]"

curl --data $txs $e:7777/add_txs
\end{simplesh}

When \Wch{} encounters a transaction of this type when executing a block, it
simply write \texttt{ok} in its transaction receipt, no VM is initialized.

\begin{weakBox}
  \emoji{turtle} : In other blockchain, the simplest transaction is usually a
  ``transfer'' transaction. But there's no ``currency'' in \Wch{}. So no
  built-in transfer transaction. However, you can always implement one using
  smart contract.
\end{weakBox}

\subsection{Python Contract}
\label{sec:pyvm}

\Wch{} allows you to write smart contracts in Python.

Deploy a Python contract:

Suppose we have \texttt{/tmp/tmp.py}:
\begin{simplepy}
from typing import Any
def hi(_storage: dict[str, Any], y : int) -> int:
    _storage["x"] = _storage.get("x", 1) + y
    return _storage["x"]
\end{simplepy}
Then to deploy the contract, we can use:
\begin{simplesh}

txs='[{"from" : "01","to" : "",
 "data" : "@/tmp/tmp.py",
 "nonce" : 123,
 "type" : "python"
}]'
# jq converts the text file into valid jsonstring

e=http://localhost
curl --data $txs $e:7777/add_txs
# => "deployed_address": "0000000000000000000000000000000064d05d1b"
\end{simplesh}

Invoke a Python contract:

Suppose we have \texttt{/tmp/tmp.json}:
\begin{simplejs}
{
        "method" : "hi",
        "args" : {"y" : 122}
}
\end{simplejs}
Then to invoke the contract, we can use:
\begin{simplesh}

txs='[{
    "from" : "01",
    "to" : "'"$da"'",
    "data" : "@/tmp/tmp.json",
    "nonce" : 124,
    "type" : "python"
}]'
e=http://localhost
curl --data $txs $e:7777/add_txs
# => "hash": "c63fce1da4f0a30c0ae6c8ce332286f188d71dc13638beca74fb5939c5a79ec8",

# get the result
curl "$e:7777/get_receipt?hash=c63fce1da4f0a30c0ae6c8ce332286f188d71dc13638beca74fb5939c5a79ec8"
# {"ok":true,"result": 123}
\end{simplesh}

\emoji{parrot} : Hey, the python contract is much friendlier!

\emoji{turtle} : Yeah, it's designed to be so. To learn more,
see \cref{sec:how-to-py}.

\subsection{Start a Raft Cluster}
\label{sec:tut-raft}

\subsection{Start a Pbft Cluster}
\label{sec:tut-pbft}

\subsection{Production Cryptography}
\label{sec:tut-serious-crypto}